\section{Resultados y discusión}

Se han obtenido resultados según los objetivos planteados en el apartado en el que se establecen estos. Han sido logrados mediante la búsqueda de información en diversas fuentes y la selección, a su vez, de la información relevante en el tema a desarrollar.

En cuanto a la información encontrada, se ha buscado en fuentes de datos fiables, siendo ésta de buena calidad en su mayoría. El mayor problema en la búsqueda de información ha sido la búsqueda en las bases de datos de aquellos documentos o artículos en los que, después de haberlos encontrado, no aparece el texto, mostrándose sólo el nombre del autor y el título, sin posibilidad de disponer de él para adquirir información. Este problema ha sido contrarrestado parcialmente gracias a la página web de la biblioteca de la Universidad del País Vasco / Euskal Herriko Unibertsitatea.

En algunos casos la información se contradice, es el caso, por ejemplo, de la localización de los clavos de las extremidades superiores en la crucifixión. En este tema existen experimentos en los que se muestra que un crucificado no podría mantenerse sujeto a la cruz mediante la inserción de los clavos en las palmas de las manos, mientras que otros estudios dicen que si es posible al tener ambos pies también sujetos y no colgando. Aún así, y después de leer la limitada bibliografía que existe acerca de ello parece que la hipótesis más extendida es la de que los clavos se encontraban a la altura de las muñecas.

Sobre este mismo tema acerca de la crucifixión de Cristo y las causas de su muerte no existen numerosos artículos que lo amplien, sí que he podido encontrar varios en los que se habla de las distintas hipótesis que se manejan, no sacando en claro cuál es la más fidedigna. Por ello he dejado constancia de todas aquellas posibilidades de acuerdo con el estudio científico tanto de lo mencionado en la Biblia como de otras crucifixiones conocidas.

En otros apartados del trabajo no ha habido tanto problema para la obtención de información, la cual, sin embargo, ha tenido que ser resumida y acotada al tema relevante. Es el caso de la historia de la anatomía en la que existen numerosos documentos, sobre todo acerca de Andrea Vesalio.

Refiriéndome al tema principal del trabajo, he de admitir que la anatomía de superficie no es un tema del que haya podido obtener mucha información. Empezando por la propia definición del término: tras haber leído una cantidad no muy extensa de definiciones, ninguna que me complaciese del todo, he decidido hacer una propia basándome en las anteriormente leídas.

Para analizar la anatomía de superficie de las obras pictóricas analizadas, al final he tenido que recurrir a varios libro que se centraba en esta desde un punto de vista totalmente médico, habiéndole añadido yo la parte artística según mi propio conocimiento e ingenio. Sin embargo la parte artística de las obras está perfectamente documentada en diferentes libros y páginas de internet.

En general y tras diversas dificultades %que he encontrado por el camino, 
creo que he llegado a cumplir la mayor parte de los objetivos formulados, desarrollando un TFG interesante y muy trabajado.