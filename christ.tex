\section{Muerte de Cristo}
La muerte de Cristo es una tema muy recurrido en la obra pictórica. En los siglos inmediatamente posteriores a la crucifixión no representaban tal cual al Cristo en la cruz, sino que lo hacían mediante otros símbolos y más tarde exclusivamente la cruz. Fue, sobre todo, a partir de la Edad Media cuando empezó a ser frecuente la imagen del Cristo crucificado.

Por otra parte, la iconografía cristiana varía de forma y estilo según las percepciones de la época artística en la que se desarrolla, en la que predomina un estilo determinado, pero en todas ellas ha tenido gran relevancia. De un joven imberbe se pasa a un hombre con barba. Ambas representaciones confluyen, pero artísticamente al final persiste la segunda. También se pasa de una visión de un Cristo triunfal a una visión más humana y menos divina de Cristo en el suplicio de la crucifixión.

Puesto que fue el emperador Constantino quien abolió la crucifixión en el siglo IV d.C, y no hay documentación ni representaciones acerca de la crucifixión hasta varios siglos después, es difícil saber cómo era en la época romana esta pena capital, utilizada únicamente contra los peores delincuentes.

Se sabe que aunque ha habido distintas formas de crucifixión \footnote{Anexo}, los romanos del periodo comprendido alrededor del nacimiento y la muerte de Cristo utilizaban la cruz \textit{commisa}, la más frecuentemente representada, o la cruz \textit{immissa}, en forma de T. Éstas estaban formadas por dos tablones de madera: un poste vertical que se insertaba en el suelo y un travesaño horizontal denominado patibulum. A veces en la mitad del poste vertical se insertaba un bloque de madera que servía de asiento. También se utilizaba, aunque probablemente posteriormente al tiempo en el que vivió Cristo, un tablón que colocado a la altura de los pies ejercía de apoyo para éstos. Aunque había distintas alturas para las cruces, la altura que solían tener era de unos dos metros, por lo que las representaciones que podemos observar de crucifixiones, en su mayoría, podrían estar equivocadas.

Los Romanos utilizaban cuerdas o clavos para sujetar a los individuos crucificados y la crucifixión variaba según la región y la inventiva del verdugo.

La creencia más extendida actualmente es la de que los clavos eran introducidos a la altura de la muñeca, entre el cúbito y el radio y justo entre estos y los carpos o entre las dos filas de huesos metacarpianos. Se ha llegado a esta conclusión tras haber encontrado hallazgos arqueológicos de esta época que concuerdan con esta teoría. Anteriormente se creía que los clavaban en las manos por que así lo establece Juan en su Evangelio (Jn 20, 20-29), pero antiguamente la muñeca se encontraba en lo que se denominaba ``mano", que abarcaba hasta el brazo, lo que ha podido conducir a error durante siglos.

La práctica más utilizada para la inserción de los clavos de los pies era hacerlo entre el segundo y tercer metatarsiano y juntando ambos pies, utilizado para ello un solo clavo. Otras prácticas eran la utilización de cuerda para atar los pies a la cruz, y la fijación de éstos por separado mediante dos clavos.

En las representaciones pictóricas se emplea tanto la imagen de Cristo crucificado mediante tres clavos como mediante cuatro.

% Enciclopedia moderna, 11: diccionario universal de literatura, ciencias, artes, agricultura, industria y comercio. Francisco de Paula Mellado (Pág:805-811)
% http://www.frugalsites.net/jesus/crucifixion.htm
% http://mb-soft.com/believe/text/crucifix.htm
% LA CRUCIFIXIÓN Laura RODRÍGUEZ PEINADO (en documentos del TFG)
% http://www.shroud.com/bucklin2.htm

Si existe debate acerca de si Jesús murió verdaderamente en la cruz \footnote{Algunos autores como Margaret y Trevor Lloyd Davies, los mayores defensores, consideran la posibilidad de que Jesucristo no muriese verdaderamente en la cruz. Otro ejemplo de esta hipótesis se encuentra reflejada en el libro \textit{42 días, Análisis forense de la crucifixión y la resurrección de Jesucristo}, de Miguel Lorente.}, mayor es el debate existente sobre la presunta causa de su muerte.

Existen varias teorías al respecto: 1) Embolia Pulmonar 2) Rotura cardíaca 3) Trauma de suspensión 4) Asfixia 5) Herida perforante fatal 6) Shock 7) Síncope fatal 8) Arritmia cardíaca 9) Coagulopatía inducida por traumatismos.

Todas estas hipótesis han sido analizadas en profundidad, no llegando, sin embargo, a la conclusión acerca de cuál es la real. Lo que está claro, sin embargo, es que Cristo padeció terribles sufrimientos, los cuales en conjunto pudieron conducirle a su muerte, que se produjo hacia las tres de la tarde tras horas en la cruz.

% The Search for the Physical Cause of Jesus Christ's Death Author: W. Reid Litchfield Categories: God and Jesus Christ, Science Journal: 37:4 (en documentos de TFG)
% The crucifixion of Jesus: Review of hypothesized mechanisms of death and implications of shock and trauma-induced coagulopathy (en refworks)