ANATOMIA DE SUPERFICIE

CABEZA

Cara:

Huesos de la cara: la frente esta formada por la convexidad lisa del hueso frontal. Sus bordes inferiores (arcos supraciliares) dan lugar al borde superior de cada órbita.Estos dos arcos se unen con un puente denominado gabela. La depresión que hay debajo (entre la frente y la nariz) es el nasión.
El hueso maxilar articula con el frontal formando la cara interna de la órbita y junto con el hueso lacrimal aloja al aparato lacrimal.
La nariz  está formada por los huesos nasales que articulan entre sí y con los huesos frontal (por arriba) y maxilar (por el lado). Las suturas internasales y frontonasales se cruzan en el nasión.Además la parte más promiinente de la nariz esta formada por el cartílago nasal que separa las fosos nasales con el septo nasal
El borde externo de la órbita está formado por los huesos frontal y cigomático. Este último forma también la prominencia de la mejilla y el arco cigomático (junto con la apófisis cigomática del hueso temporal)y junto con el hueso maxilar el borde inferior de la órbita.
El borde inferior del maxilar presenta los alveolos para los dientes superiores, así como el hueso de la mandíbula contiene los de los huesos inferiores.

Músculos de la cara: Los músculos de la cara contribuyen mayoritariamente a dotar a la cara de expresión facial. Estos músculos se disponen como esfínteres (músculo orbicular del ojo y músculo orbicular de la boca) y dilatadores alrededor de la órbita, nariz y boca. Además existe el músculo frontal que se une con la apófisis occipital y la platisma que se une al tejido subcutáneo del cuello y del toráx superior.

El resto de la cabeza está formada por la bóveda craneal. La cara lateral de la bóveda la componen los huesos frontal parietal occipital y temporal.
Las apófisis de los huesos frontal y cigomático (que forman el borde externo de la órbita)forman el borde anterior de la fosa temporal. La línea temporal da inserción a la fascia del músculo temporal  (puede sentirse al masticar)que inferiormente se inserta en el arco cigomático.
El múculo masetero va desde la cara externa posteroinferior de la mandíbula al borde inferior del arco cigomático (se acentúa al apretar los dientes).
Detrás de la oreja se encuentra la apófisis mastoides (recubierta por el esternocleidomasoideo y el músculo digástrico).

Mandíbula: cuerpo, rama y ángulo. El cóndilo mandibular se palpa con dificultad con la mandíbula cerrada pues está cubierto por el ligamento lateral, pero se hace visible al abrir la mandíbula.

Órbita-globo ocular: limitada anteriormente por los párpados superior e inferior que estań unidos en sus extremos medial y lateral, limitando la hendidura palpebral. En sus bordes se encuentran las pestañas dispuestas en dos o tres hileras irregulares. En el lado medial se encuentra la carúncula lacrimal, que se une al saco lacrimal mediante los canalículos lacrimales (no visibles). Los párpados estan rodeados por el músculo orbicular y el superior se eleva por la acción del músculo elevador del párpado. Los movimientos oculares se deben a los músculos extraoculares.

Cavidad oral: Es el principio del tubo digestivo y su membrana mucosa comienza en la cara anterior de los labios (parte rosada). Contiene lengua, las arcadas dentarias con encías y dientesy recibe los orificios de salida de las glándulas salivares. Existen el paladar duro, formado por los huesos maxilar y palatino y el paladar blando que termina en la úvula. La lengua descanas sobre el suelo de la boca. El frenillo une la cara inferior de la lengua con el suelo de la boca.

Oreja: formada por piezas de fibrocartilago dispuestas de forma irregular a las que se une la piel. Hélix, antihélix, trago, antitrago, lóbulo, orificio auditivo externo.


CUELLO:

Visión anterior: Delimitado superiomente por la mandíbula (cuerpo y ángulos), inferiormente por la hendidura esternal del manubrio y las clavículas. Las clavículas articulan con el esternón (unión esternoclavicular) y con el acromion (unión acromioclavicular).
El músculo esternocleidomastoideo se inserta en la apófisis mastoides y tercio externo de la línea nucal superior por su límite superior y en el esternón y la clavícula por su parte inferior. Los bordes anteriores de la parte esternal y de la parte clavicular forman una V prominente cuando se contraen simultáneamente (cuando se eleva la cabeza en posición de decúbito supino).

Laringe: Se sitúa en la línea media del cuello. Puede verse cómo se eleva durante la deglución. En el borde superior se encuentra el hueso hioides (forma de U) a nivel de la cuarta vértebra cervical. El cartílago tiroides forma una prominencia en la línea media que es más evidente en el varón (angulo entre sus alas 90º, mientras en la mujer 120º). Las cuerdas vocales se insertan en su parte posterior y los músculos se unen al hueso con el hioides superiormente y al dorso del manubrio inferiormente. El cricoides forma parte del borde inferior de la laringe y es el único anillo completo de cartílago en la vía respiratoria (los demás están incompletos por su cara posterior). La glándula tiroides está formada por varios lóbulos y un itsmo que los une (línea media anterior al segundo o tercer cartílagos traqueales). La glándula no se aprecia, sin embargo si aumenta de tamaño (bocio) se puede ver cómo se mueve con la laringe. 

Región submandibular: El triángulo anterior (esternocleidomastoideo, rama mandibular y línea media) contiene otros tres triángulos formados por dos músculos insertados en el hueso hioides. El triángulo digástrico formado por el vientre anterior y posterior del músculo digástrico (unidos por un tendón al hueso hioides) y el borde inferior de la mandíbula. El triángulo carotídeo formado por el vientre posterior del músculo digástrico, el músculo omohioideo y el esternocleidomastoideo. Y el triángulo muscular delimitado por el omohioideo, el vientre anterior del digástrico y la línea media. El triángulo digástrico contiene la glándula submandibular que descansa sobre y alrededor del borde posterior del músculo milohioideo (de la lengua al hioides). 

Visión lateral:Delimitado superiormente por cuerpo, ángulo y ramas de la mandíbula y la articulación temporomandibular Inferiormente está delimitada por la clavícula y el acromion que forman el vértice del hombro (articulación acromioclavicular).
El triángulo posterior formado por esternocleidomastoideo, clavícula y músculo trapecio. En un individuo delgado con la cabeza vuelta y flexionada al lado contralateral se pueden ver el omohioideo cruzando el triángulo y el músculo escaleno anterior descendiendo.

Visión posterior: Delimitada superiormente por la línea nucal superior del hueso occipital que termina medialmente en la protuberancia occipital externa y lateralmente en la apófisis mastoides.
El ligamento nucal se encuentra en la línea media y se inserta en el hueso occipital y en las apófisis espinosas de las vértebras cervicales. Son visibles las apófisis espinosas de las dos o tres vertebras cervicales inferiores, así como todas las torácicas. La apófisis espinosa de la séptima vertebra cervical  y la primera torácica son las más evidentes. También son visibles, el acromion y la espina escapular.
El músculo trapecio se inserta superiormente en el tercio medio de la línea nucal superior.


TÓRAX

Pared torácica anterior: Desde las clavículas hasta el borde costal inferior. Formado por el esternón, las costillas y cartílagos costales y los músculos relacionados.
El esternón está formado por tres partes: el manubrio, el cuerpo y la apófisis xifoides. Se puede observar la escotadura external sobre la cara superior del manubrio. El manubrio y el cuerpo esternal son palpables y están unidos por una articulación cartilaginosa. Las costillas (visibles en individuos delgados) se cuentan desde esta localización: el cartilago de la segunda costilla se articula a cada lado de la articulación manubrioesternal. La apófisis xifoides se encuentra cubierta por el recto anterior del abdomen, por lo que no se aprecia superficialmente, cuando crece y se aprecia se debe considerar anormal.
La caja costal está formada por doce pares de costillas (parte ósea posterolateral y parte anterior catilaginosa). La primera costilla no de palpa fácilmente, ya que está por debajo del pectoral mayor y de la clavícula. Las siete primeras costillas (verdaderas) articulan directamente co el esternón, de la octava a la décima (falsas) Articulan con la costilla superior y las costillas once y doce son las flotantes. Las seis últimas costillas forman el reborde costal inferior. Los espacios intercostales están ocupados por los músculos intercostales, insertados en las costillas adyacentes.
La clavícula se observa en toda su longitud, en su cara interna articula con  el esternón (articulación esternoclavicular).
El músculo pectoral mayos se inserta en el borde interno de la clavícula, en los cartílagos de las cinco costillas superiores y en el húmero, formando el pliegue axilar anterior. El pectoral menor está cubierto por el mayor por lo que no se percibe superficialmente.
El músculo deltoides forma el contorno redondeado del hombro que recubre la articulaciom del hombro, insertándose en la clavícula y la escápula.
Los músculos de la pared abdominal se insertan en las costillas inferiores y en los cartílagos costales.
Todas las apófisis espinosas de las vértebras torácicas se palapan en la línea media. La espina y acromion de la escápula son visibles durante los movimientos del brazo. La escápula externamente se articula con la clavícula y el húmero, pero carece de inserción ósea medial. La moyoría de la caja costal no es visible desde la cara posterior, estando cubierta por los músculos erectores de la columna y la escápula y sus inserciones musculares.

Pared torácica lateral: Limitada inferiormente por el borde inferior de la caja costal y superiormente por la axila.
El músculo serrato anterior se inserta en el borde anterior interno de la escápula y en las ocho costillas superiores, formando ocho facículos que pueden verse en una persona delgada junto con sus interdigitaciones con el músculo oblicuo externo en las cuatro costillas medias.
La glándula mamaria femenina se extiende de la segunda costilla a la séptima y desde el borde externo del esternón hasta la pared axilar anterior. Descansa sobre los músculos  pectoral mayor, serrato anterior y oblicuo externo. El tamaño de la mama y la altura del pezón son variables en cada mujer. En el varón el pezón se encuentra en el cuarto espacio intercostal.
Axila: Es una cavidad rellena de grasa que separa el brazo del tórax. Su pared interna está formada por las seis costillas superiores y las paredes anterior(pectoral mayor) y posterior (músculo redondo mayor) convergen lateralmente sobre la porción superior de la diáfisis humeral.


ABDOMEN Y PELVIS

Pared abdominal anterior: Delimitada superiormente por el reborde costal inferior e inferiormente por la sínfisis del pubis, la tuberosidad púbica, el ligamento inguinal, la espina ilíaca anterosuperior y la cresta ilíaca (de medial a lateral). Los músculos abdominales forman una vaina fibrosa que contiene los músculos a cada lado de la línea media, las vainas de los dos lados se unen en la línea media formando un rafe fibroso, conocido como línea alba. Pueden apreciarse tres líneas transversales por encima del ombligo debidas a la inserción del recto anterior de abdomen a la vaina anterior, denominadas inserciones tendinosas (en un sujeto musculado). También se puede apreciar la línea semilunar que marca el borde lateral del recto. El músculo mas superficial de los músculos laminares del abdomen es el oblicuo externo que se inserta superiormente en las ocho costillas inferiores y tiene interdigitaciones con el serrato anterior en las cuatro costillas medias. Inferiormente se inserta en la sínfisis del pubis, tuberosidad púbica, espina ilíaca anterosuperior y mitad anterior de la cresta ilíaca y entre la espina y la tuberosidad púbica se vuelve sobre sí mismo formando el ligamento inguinal.

Región posteroinferior del tronco: Se encuentran las costillas inferiores, la cresta iĺíaca, las espinas ilíacas posteriores, superiores e inferiores, las apófisis espinosas lumbares y la cara posterior del sacro.
Curvas de la columna y movimientos del tronco: Las curvas y los discos intervertebrales confieren cierta elasticidad a la columna. La mayoría de los músculos de la espalda mantienen la posición erecta del cuerpo. Estando de pie, en reposo, el centro de gravedad se encuentra en la segunda vértebra del sacro. Los movimientos del cuerpo desplazan el centro de gravedad hacia delante y se requiere una gran masa de músculos posteriores para volver a la posición erecta. Flexión del tronco: Recto abdominal y músculos paravertebrales, extensión: músculos largos de la espalda, flexión lateral: músculos oblicuos y cuadrado lumbar, rotación:oblicuo interno y externo. Alteraciones en la columna vertebral pueden dar lugar a escoliosis o cifosis, que a su vez pueden coexistir (cifoescoliosis).

Región inginal, periné, escroto y pene: El músculo oblicuo externo se enrolla sobre sí mismo formando el ligamento inginal (va desde la espina ilíaca anterosuperior hasta la tuberosidad púbica). En el varón forma un orificio a través de la pared abdominal por el que el testículo pasa descendente para llegar al escroto. Los testículos pueden palparse dentro del escroto, al igual que los polos superior e inferior de l epidídimo. El cuerpo del pene termina ensanchándose en el glande del pene, que contiene un orificio externo llamado uretra. Recubriendo el glande se encuentra la piel del prepucio que se extirpa en la circuncisión.

Periné femenino: La apertura de la vagina esta rodeada por los labios mayores y los menores. Los labios mayores se unen anterosuperiormente alrededor del clítoris. La uretra desemboca entre el clítoris y la vagina. El ano esta situado en la linea media, alineado con las tuberosidades isquiáticas, anterior al coxis.


MIEMBRO SUPERIOR

Visión anterior del hombro y del brazo: la cintura escapular une el hueso húmero con el tronco. Está formada por la escápula y la clavícula y músculos que la sujetan. La clavícula articula medialmente con el esternón y la primera costilla (unión esternoclavicular).Lateralmente articula con la escápula (unión acromioclavicular).

El deltoides da forma al contorno del hombro cubriendo la epífisis proximal del húmero, insertándose en la clavícula, el acromion y la espina de la escápula. Lateralmente se inserta en el tubérculo deltoideo de la cara lateral del húmero. Es el principal abductor del brazo.

El músculo pectoral mayor tiene dos fascículos mediales. El clavicular que inserta en los dos tercios mediales de la clavícula. El esternocostal inserta en el esternón y en los seis cartílagos costales superiores. Lateralmente ambos fascículos convergen en un tendón  que se inserta por debajo del deltoides en el labio lateral del surco bicipital del húmero. Juntos forman la mayor parte de la cara anterior de la axila. Es un potente adductor. Junto al dorsal ancho hace prominencia al presionar las manos sobre las caderas.

La apófisis coracoides da inserción a los músculos coracobraquial, cabeza corta del bíceps y pectoral menor. Puede palparse un centímetro por debajo de la clavícula bajo las fibras mediales del deltoides.

El bíceps forma una prominencia en la parte anterior del brazo. Su cabeza corta inserta en la apófisis coracoides y la larga en la cara superior de la fosa glenoidea de la escápula. Distalmente se inserta en la tuberosidad bicipital del radio. Flexiona el codo y es también un supinador del antebrazo.

Visión posterior del hombro y del brazo: El acromion y la espina de la escápula son visibles.

El trapecio forma una ancha lámina triangular que se inserta medialmente en la mitad medial de la línea nucal superior del hueso occipital, el ligamento nucal, las apófisis espinosas y los ligamentos interespinosos de las vértebras cervicales inferiores y de todas las torácicas. Lateralmente se inserta en el tercio externo de la clavícula, el acromion y la espina de la escápula. Se encarga de un amplio rango de movimientos escapulares.

El músculo dorsal ancho también tiene una inserción medial ancha: apófisis espinosas lumbares,fascia lumbar y mitad posterior de la cresta ilíaca. También se inserta en el ángulo inferior de la escápula y ayuda a impedir que este ángulo sobresalga de la pared torácica durante los movimientos del hombro. Su tendón estrecho envuelve al músculo redondo mayor en la pared posterior de la axila. Es un potente adductor del brazo.

La parte inferior del pliegue axilar está formada principalmente por el redondo mayor, que va desde el borde lateral de la escápula hasta el surco bicipital del húmero. El borde lateral de la escápula puede percibirse a través de la masa del redondo mayor.

Los músculos infraespinoso (cara posterior de de la escápula por debajo de la espina) y subescapular (cara anterior de la escápula) forman el pliegue axilar superior. El músculo supraespinoso se inserta en la cara posterior de la escápula sobre la espina, pasa inferior al acromionhacia la tuberosidad mayor del húmero. Es importante para iniciar la abducción. Estos músculos junto con el redondo mayor refuerzan la articulación del hombro anterior, superior y posterior.

Movimientos de la escápula y de la articulación del hombro: los desplazamientos del miembro superior se producen gracias a los movimientos de la escápula y la clavícula.

La escápula se eleva al encogerse los hombros, mediante las fibras superiores del trapecio y el músculo elevador de la escápula. Forma parte del relieve muscular de la nuca en visión posterolateral. La escápula desciende por acción de la gravedad y del serrato anterior y el pectoral menor (estos dos también lo mueven hacia delante). La retracción de la escápula se realiza mediante el el trapecio y el romboides (profundo). La rotación externa de la escápula la realizan el serrato anterior y el trapecio y la rotación interna los músculos elevador de la escápula, pectoral menor y romboides. La flexión del hombro la realizan el facículo clavicular del pectoral mayor, las fibras anteriores del deltoides y el coracobraquial, mientras que la extensión la realizan las fibras posteriores del deltoides y, si se pate de la flexión, también el pectoral mayor y el dorsal ancho. La abducción la inicia el supraespinoso y la continúa el deltoides. La rotación interna: pectoral mayor, deltoides, y dorsal ancho. Rotación externa: deltoides, redondo menor e infraespinoso.

Fosa cubital: limitada superiormete por la línea que va desde la epitróclea hasta el epicóndilo del húmero. Lateralmente limitada por el músculo braquiorradial y medialmente por el pronador redondo.
El músculo braquial tiene una inserción en la cara anterior del húmero y su tendón pasa a la apófisis coronoides del cúbito, dificultando la palpación de los huesos en la cara anterior de la articulación del codo. El tendón del bíceps pasapor el medio de la fosa hasta la tuberosidad bicipital del radio. La aponeurosis bicipital va desde la cara medial del tendón a la superficie subcutánea del cubital. Esta dificulta la palpación de la arteria braquial que pasa profunda por el vértice de la fosa cubital donde se divide en arterias radial y cubital. Las venas cefálica, basílica y cubital discurren superficiales y son habitualmente visibles.

Visión anterior del antebrazo: El músculo pronador se inserta sobre la cara anterior de la epitróclea y en la región central del radio. Es un pronador del antebrazo. Los tendones de los músculos flexor carpi radialis, palmar mayor, flexor superficial de los dedos y flexor carpi ulnaris pasan a la mano. Son también flexores débiles del codo.
El músculo braquiorradial se inserta proximalmente  en los dos tercios superiores de la cresta supracondílea del húmero y distalmete en la cara lateral de la epífisis distal del radio. Este músculo prona o supina el antebrazo.

Visión posterior del codo y el antebrazo: Los huesos que forman la articula ción del codo son visibles por la cara posterior. Se observa la epitróclea (medialmente), el epicóndilo (lateralmente), asi como el borde posterior del cúbito, subcutáneo en toda su longitud y el olécranon donde se inserta el múculo tríceps que es la principal masa muscular del brazo posterior. La cabeza del radio envainada en el ligamento anular es palpable y puede ser visible.

Movimientos de las articulaciones del codo y radio-cubital: El codo es flexionado por los músculos bíceps y braquial, ayudados por el braquiorradial y los músculos del origen del flexor común (flexor carpi radialis, palmar mayor, flexor superficial de los dedos y flexor carpi ulnaris). La extensión la realiza el tríceps junto con la contribución de los músculos del origen del extensor común.
En posición anatómica la posición del antebrazo está en total supinación. La epífisis distal del radio puede hacer la rotación interna anterior al cúbito , a lo largo de 180 grados, dejando el dorso de la mano anterior y la palma posterior, esta es la posición de pronación completa. El bíceps y el supinador supinan y el pronador redondo y el pronador cuadrado pronan.
En la posición anatómica la apófisis estiloides del radio y del cúbito son palpables e incluso visibles. En completa pronación la apófisis estiloides radialsigue siendo palpable, pero ahora medialmente. La apófisis estiloides del cúbito sufre un desplazamiento lateral y edja de ser palpable. En esta posición el hueso más fácilmente palpable es la cabeza del cúbito.

Visión anterior de la muñeca y de la mano: cuando se flexiona la muñeca contra resistencia, tres tendones(flexor carpi radialis, palmar mayor y flexor carpi ulnaris) sobresalen. El flexor carpi raialis se inserta distalmente en la base de los metacarpos 2 y 3, el flexor carpi ulnaris de inserta en el pisiforme y de allí en el ganchoso y en la base del quinto metacarpiano.
El retináculo flexor es una banda fibrosa cuadrada que va medialmente del pisiforme al gancho del ganchoso y lateralmente del escafoides al borde del trapecio. El pisiforme es visible, así como el tubérculo del escafoides.

Tabaquera anatómica: es una depresión en la cara lateral de la muñeca que se acentúa al extender el pulgar. Está limitada anteriormente por los tendones del abductor largo del pulgar y del extensor corto del pulgar y posteriormente por el extensor largo del pulgar. 
El abductor largo del pulgar se inserta distal al primer metacarpiano, el extensor corto del pulgar en la falange proximal del pulgar y el extensor largo del pulgar en la falange distal del primer dedo. La vena cefálica cruza la fosa superficialmente.

Visón dorsal de la muñeca y de la mano: la epífisis distal del radio es visible. Las articulaciones metacarpofalángicas de segunda a cuarta son facilmente visibles. El primer metecarpiano es mucho más móvil que los otros y su articulación carpometacarpiana es facilmente palpable (visible no mucho). Las articulaciones interfalángicas son palpables alrededor de su circunferencia.
Los tendones del extensor de los dedos están sujetos al hueso mediante el retináculo extensor que pasa por encima oblícuo desde la cara lateral del radio hasta la porcion medial del cúbito. Cada tendón del extensor de los dedos forma un triángulo (la expansión dorsal) sobre la articulación metacarpofalángica. Los músculos lumbricales e interóseos correspondientes se insertan en la baso de esta expansión. Otros dos tendones pueden verse en el dorso de la muñeca, el extensor de meñique que discurre lateral al tendón largo del quinto dedo y el extensor del índice que discurre medial al tendón largo del dedo índice. Ambos insertan en la expansión dorsal.

Movimientos de la muñeca y de la mano: la flexión se realiza en la articulación mediocarpal (entre las dos hileras de carpos) desarrollada por los músculos flexor carpi radialis y flexor carpi ulnaris ayudados por los flexores largos de los dedos. La extensión se realiza en la articulación radiocarpal por los músculos extensor carpi radialis longus y brevis y el extensor carpi ulnaris ayudados por los extensores largos de los dedos. La abducción es en la articulación mediocarpal por los músculos flexor carpi radialis y extensor carpi radialis longus y brevis. La adducción en la articulación radiocarpal mediante el flexor carpi ulnaris y el extensor carpi ulnaris.

Eminencia tenar: es el abultamiento lateral de la palma de la mano y está formada por los músculos cortos del pulgar: los músculos abductor, floxor y oponente del pulgar que se insertan en el tubérculo del escafoides, la cresta del trapecio y el retináculo flexor. El abductor corto y el flexor corto del pulgar insertan en la falange proximal del pulgar y el oponente del pulgar en el primer metacarpiano.
La flexión se combina con la rotación interna y la realizan los flexores largo y corto y el oponente del pulgar. La extensión se combina con la rotación externa y la realizan los extensores largo y corto y abductor largo del pulgar. EL abductor corto del pulgar realiza la abducción, el adductor del pulgar la adducción y el oponente del pulgar la oposición.

Eminencia hipotenar: está formada por el abductor, el flexor y el oponente del meñique, originándose en el retináculo flexor hasta la falange proximal y el margen cubital del quinto metacarpiano. El meñique, aunque menos móvil que el pulgar, puede realizar una discreta rotación.

Músculos interóseos: las articulaciones metacarpofalángicas de 2º a 4º pueden realizar abducción y aduccion (las del pulgar solo flexión y extensión). Movimientos que se deben a la acción de los músculos interóseos ventrales (adducción) y dorsales (abducción). Los interóseos palmares se insertan en las caras palmares de los metacarpianos 1º, 2º, 4º y 5º. Los dorsales son más largos y potentes originándose por dos cabezas en los lados de los metacarpianos adyacentes. Todos se insertan en la falange proximal y en la expansión del extensor correspondiente.

Músculos lumbricales: son cuatro músculos débiles que se originan en la cara lateral de de cada tendón del flexor profundo de los dedos en la palma y se insertan en la cara lateral de de la expansión dorsal del mismo dedo. Flexionan la articulación metacarpofalángica y extienden las articulaciones interfalángicas proximal y distal, ayudados por los músculos interóseos.

La extensión de los dedos la lleva a cabo el músculo extensor largo de los dedos, ayudado por el extensor del índice y del meñique.

MIEMBRO INFERIOR

Triángulo femoral: la sínfisis y el cuerpo del pubis y la espina ilíaca antero superior de la pelvis son palpables (la última también es visible) en la cara anterior. El ligamento inguinal, que va desde la espina ilíaca a el tubérculo púbico, forma la base del triángulo femoral. El borde lateral lo forma el músculo sartorio, que inserta en la espina ilíaca y en la superficie subcutánea medial de la parte superior de la tibia. Nos permite cruzar las piernas. El borde interno es la cara medial del músculo abductor largo que va desde el cuerpo del pubis hasta la línea áspera del fémur. El techo del triángulo lo forma la fascia lata que rodea el muslo y se inserta superiormente  en el ligamento inguinal, la resta ilíaca, la rama inferior del pubis y el sacro. El suelo del triángulo está formao por los músculos iliopsoas y pectíneo (anteriores a la cadera). El iliopsoas se inserta en proximalmente en la interna del ilion y en la cara lateral de las vértebras lumbares, distalmente sobrepasa el trocánter menor del fémur. Es flexor de la cadera. El pectíneo se inserta en la rama superior del pubis y justo debajo del tendón del iliopsoas en el fémur.

Visión anterior y medial de la cadera, muslo y rodilla: la cara anterior del muslo está formada principalmente por el cuádriceps. Éste tiene cuatro inserciones proximales, el recto anterior se inserta a la espina ilíaca anteroinferior y una cabeza oblicua se inserta encima del acetábulo. Los vastos interno y externo se insertan en la línea áspera del femur y el vasto intermedio sobre la cara anterior de la diáfisis femoral. En su extremo inferior se inserta a la r´otula y se extiende sobre esta formando el "ligamento patelar" insertado en la tuberosidad tibial. Rl ligamento patelar y las fibras horizontales del vasto interno (se atrofian con la inactividad), pasando hacia la cara medial, son visibles. La rótula, la tuberosidad tibial, los cóndilos femorales medial y proximal y el borde superior de la tibia pueden palparse (incluso verse en personas delgadas).
La cara medial del muslo contiene los músculos abductores: pectíneo, abductor largo (ya descritos), el abductor corto que pasa desde la rama inferior del pubis a la línea áspera y más profundamente el abductor mayor que se inserta medialmente a lo largo de la rama isquiopubiana y tuberosidad isquiática y lateralmente a la línae áspera, cresta medial supracondilar y el tubérculo abductor del fémur. El tubérculo puede ser visible.
Los músculos abductores están cubiertos por el músculo gracilis, que pasa desde la rama inferior del pubis a la cara superomedial de la tibia. El gracilis y los abductores abducen la cadera.
El ligamento medial colateral pasa desde el epicóndilo medial del fémur a la cara superior de la diáfisis tibial. Los tendones del sartorio, del gracilis y del semitendinoso (de anterior a posterior) se insertan en en la superficie subcutánea superior de la tibia y se marcan al flexionar la rodilla en ángulo recto.

Visión lateral de la cadera, muslo y rodilla: el trocánter mayor es palpable en la cara superolateral del muslo (visible en personas delgadas). Al sentarse el peso del cuerpo recae sobre las tuberosidades del isquion y está almohadillado por el glúteo mayor y los tejidos grasos de las nalgas.
El tendón del bíceps es fácilmente accesible en la cara lateral de la rodilla cuando ésta se encuentra flexionada en ángulo recto: se puede seguir hasta su inserción en la epífisis proximal del peroné. El borde óseo de la cara lateral del condilo tibial se puede palpar y también el ligamento peroneano.

Visión posterior de la cadera, muslo y rodilla:
Región glútea: La prominencia de la nalga está formada por un gran músculo, el glúteo mayor que tiene una inserción proximal extensa: cara lateral del ilion, el sacro, el cóccix y el ligamento sacrotuberoso. LAs fibras  van hacia abajo y laterales al iliotibial y a la tuberosidad glútea del fémur. Es un potente rotador externo y extensor de la articulación de la cadera, y a través del tracto iliotibial, extiende y estabiliza la articulación de la rodilla. El tensor de la fascia lata se inserta en el cuarto anterior de la cara externa del ilion y se inserta en el tracto iliotibial, actuando junto al glúteo mayor estabilizando y extendiendo la articulación de la rodilla.
Al estar de pie sobre una pierna el lado de la pelvis que no soporta peso se eleva para evitar que la pierna libre no pise el suelo. Esta acción está realizada por los glúteos medio y menor
La prominencia muscular de la cara posterior del muslo está formada por el grupo muscular poplíteo que se origina en la tuberosidad isquiática. El semimembranoso se inserta distalmente en un surco en la cara posteromedial del cóndilo de la tibia, también en el cóndilo externo femoral y hacia abajo en la lía sólea de la tibia, formando la fascia poplítea sobre el músculo poplíteo. El semitendinoso se inserta distalmente en la superficie superomedial de la tibia y el bíceps en la cabeza del peroné. Son extensores de la cadera y flexores de la rodilla. Delimitan superiormente la fosa poplítea.
La fosa poplítea es un espacio romboidal situado por detrás de la rodilla. El borde superior está formado por los tendones de los músculos poplíteos y es palpable y el borde inferior por las cabezas de los gemelos. La fosa está cubierta por la fascia poplítea (engrosamiento de la fascia lata). El suelo de la fosa es la cara posterior del fémur, la rodilla y el músculo poplíteo situado sobre la parte superior de la tibia.
Los movimientos de la rodilla son: flexión, extensión y un pequeño grado de rotación. Flexión: músculos poplíteos, ayudados por los gemelos. Extensión: cuadriceps y los músculos del tracto iliotibial.

Visión anterior de la pierna: la tibia es subcutánea a lo largo de su car anterior, extendiéndose desde latuberosidad tibial hasta el maleolo medial. De los cuatro músculos del compartimento anterior, sólo el tibial anterior se inserta en la tibia, su tendón distal pasa hacia la cuña medial y la base del primer metatarsiano. La inserción proximal del extensor largo del dedo gordo va desde la mitad medial del peroné, la membrana interósea y el extensor de los dedos, los dos tercios superiores del peroné y el tabique intermuscular adyacente, los tendones pasan hacia la base de las falanges distales de los dedos, los del extensor largo de los dedos pasan a los cuatro dedos externos, formando expansiones (similares a mano). El peroneus tertius se inserta en la parte inferior del peroné y el cuerpo del quinto metatarsiano. Estos músculos dorsiflexionan el tobillo, y además, el extensor largo del dedo grueso y el extensor largo de los dedos dorsiflexionan los dedos de los pies.
Los dos retináculos extensores discurren a través de los tendones del grupo muscular anterior en la parte inferior (frente al tobillo). El superior es un engrosamiento de la fascia profunda entre la tibia y el peroné. El inferior (bifurcado) pasa desde la parte superior del calcáreo atravesando el tobillo: la banda superior pasa hacia el maleolo interno y la inferior se fusiona con la fascia plantar.

Visión anterior del tobillo y del pie: la hilera distal del tarso, los metatarsianos y las falanges son palpables entre los tendones dorsales (que son visbles).El extensor corto de los dedos es el único músculo situado en el dorso del pie y se extiende desde la parte anterosuperior del calcáneo. El medial de sus tendones se unserta en la base de la falange proximal del dedo gordo, los otros tendones se unen a la cara lateral de los tendones del extensor de los dedos 2º,3º y 4º. Este músculo dorsiflexiona los dedos.

Visión medial de la pierna: la cara medial de la tibia es subcutánea. La epífisis distal del hueso se prolonga en el maleolo interno (visible). La cara posteromedial del calcáneo, el sustentaculum tali, el tubérculo del navicular y la cara medial de los metatarsianos y falanges son palpables a lo largo de la cara medial del pie.

Visión lateral de la pierna: la cabeza del peroné se palpa por debajo e la articulación de la rodilla, pero su diáfisi está rodeada por vientres musculares. La epífisis distal del hueso se prolonga en el maleolo externo (1 cm más abajo que el interno - visible). La cara lateral del calcáreo, su tubérculo peroneal, la cabeza del astrálogo y la base del quinto metatarsiano son fácilmente palpables sobre esta cara. Los dos músculos del compartimento lateral son el peroneo largo y el peroneo corto que se insertan respectivamente en los dos tercios superiores y el tercio inferior de la cara lateral del peroné. El tendón peroneo largo se inserta en en el primer metatarsiano, el corto  en la tuberosidad del quinto metatarsiano. Los dos músculos son eversores y flexores plantares.

Visión posterior de la pierna: el volumen de la pantorrilla está formado por los músculos gemelos y sóleo, que se unen inferiormente para formar el tendón calcáneo e insertarse en el medio de la cara posterior del calcaneo. El gemelo es superficial al sóleo y sus cabezas interna y externa se insertan en los cóndilos femorales respectivos. El sóleo protruye a cada lado del gemelo en la pantorrilla. Son potentes flexores del pie.

Planta del pie: la piel situada sobre las áreas de apoyo del talón y la parte anterior de la planta es gruesa y adherida firmemente a la fascia y su aponeurosis central de la planta del pie. La grasa subcutánea da sensación de almohadillamiento.La cara posterior del calcáneo y las cabezas de los metatarsianos son palpables a través de este almohadillamiento, pero los huesos restantes están profundos a los músculos cortos. Un hueso sesamoideo se puede palapar a cada lado de la cabeza del primer metatarsiano. El modelo de los músculos del pie es similar al de la mano, pero no existen músculos oponentes y el flexor accesorio no tiene equivalente en el miembro superior. Éste se inserta en el calcáneo posteriormente y en la cara lateral de los tendones del flexor del dedo grueso y del flexor largo de los dedos anteriormente. Los tendones del flexor largo pasan a través de la planta para alcanzar su inserción en las falanges distales, flexionando los dedos, ayudado por músculos cortos que también pueden provocar una extensión en abanico de los dedos. El tendón del tibial posterior da lengüetas para otros huesos tarsianos y metatarsianos. Los músculos cortos actúan con los tendones del flexor largo para mantener los arcos plantares. Los arcos se dividen en medial, longitudinal, lateral y transverso.
La dorsiflexión del pie la producen el músculo tibial anterior y otros músculos del compartimento extensor de la pierna. La flexión plantar la realizan fundamentalmente el gemelo y el sóleo. La inversión está producida por los tibiales anterior y posterior y limitada por la tensión de los peroneos largo y corto y el ligamento interóseo astragalocalcáneo. El movimiento se incrementa en la flexión plantar. La eversión está producida por los músculos peroneos y limitada por los músculos tibiales anterior y posterior y el ligamento medial de la articulación del tobillo. El movimiento se incrementa en la dorsiflexión.