ANATOMIA DE SUPERFICIE

CABEZA

Cara:

Huesos de la cara: la frente esta formada por la convexidad lisa del hueso frontal. Sus bordes inferiores (arcos supraciliares) dan lugar al borde superior de cada órbita.Estos dos arcos se unen con un puente denominado gabela. La depresión que hay debajo (entre la frente y la nariz) es el nasión.
El hueso maxilar articula con el frontal formando la cara interna de la órbita y junto con el hueso lacrimal aloja al aparato lacrimal.
La nariz  está formada por los huesos nasales que articulan entre sí y con los huesos frontal (por arriba) y maxilar (por el lado). Las suturas internasales y frontonasales se cruzan en el nasión.Además la parte más promiinente de la nariz esta formada por el cartílago nasal que separa las fosos nasales con el septo nasal
El borde externo de la órbita está formado por los huesos frontal y cigomático. Este último forma también la prominencia de la mejilla y el arco cigomático (junto con la apófisis cigomática del hueso temporal)y junto con el hueso maxilar el borde inferior de la órbita.
El borde inferior del maxilar presenta los alveolos para los dientes superiores, así como el hueso de la mandíbula contiene los de los huesos inferiores.

Músculos de la cara: Los músculos de la cara contribuyen mayoritariamente a dotar a la cara de expresión facial. Estos músculos se disponen como esfínteres (músculo orbicular del ojo y músculo orbicular de la boca) y dilatadores alrededor de la órbita, nariz y boca. Además existe el músculo frontal que se une con la apófisis occipital y la platisma que se une al tejido subcutáneo del cuello y del toráx superior.

El resto de la cabeza está formada por la bóveda craneal. La cara lateral de la bóveda la componen los huesos frontal parietal occipital y temporal.
Las apófisis de los huesos frontal y cigomático (que forman el borde externo de la órbita)forman el borde anterior de la fosa temporal. La línea temporal da inserción a la fascia del músculo temporal  (puede sentirse al masticar)que inferiormente se inserta en el arco cigomático.
El múculo masetero va desde la cara externa posteroinferior de la mandíbula al borde inferior del arco cigomático (se acentúa al apretar los dientes).
Detrás de la oreja se encuentra la apófisis mastoides (recubierta por el esternocleidomasoideo y el músculo digástrico).

Mandíbula: cuerpo, rama y ángulo. El cóndilo mandibular se palpa con dificultad con la mandíbula cerrada pues está cubierto por el ligamento lateral, pero se hace visible al abrir la mandíbula.

Órbita-globo ocular: limitada anteriormente por los párpados superior e inferior que estań unidos en sus extremos medial y lateral, limitando la hendidura palpebral. En sus bordes se encuentran las pestañas dispuestas en dos o tres hileras irregulares. En el lado medial se encuentra la carúncula lacrimal, que se une al saco lacrimal mediante los canalículos lacrimales (no visibles). Los párpados estan rodeados por el músculo orbicular y el superior se eleva por la acción del músculo elevador del párpado. Los movimientos oculares se deben a los músculos extraoculares.

Cavidad oral: Es el principio del tubo digestivo y su membrana mucosa comienza en la cara anterior de los labios (parte rosada). Contiene lengua, las arcadas dentarias con encías y dientesy recibe los orificios de salida de las glándulas salivares. Existen el paladar duro, formado por los huesos maxilar y palatino y el paladar blando que termina en la úvula. La lengua descanas sobre el suelo de la boca. El frenillo une la cara inferior de la lengua con el suelo de la boca.

Oreja: formada por piezas de fibrocartilago dispuestas de forma irregular a las que se une la piel. Hélix, antihélix, trago, antitrago, lóbulo, orificio auditivo externo.


CUELLO:

Visión anterior: Delimitado superiomente por la mandíbula (cuerpo y ángulos), inferiormente por la hendidura esternal del manubrio y las clavículas. Las clavículas articulan con el esternón (unión esternoclavicular) y con el acromion (unión acromioclavicular).

El músculo esternocleidomastoideo se inserta en la apófisis mastoides por su límite superior y en el esternón y la clavícula por su parte inferior. Los bordes anteriores de la parte esternal y de la parte clavicular forman una V prominente cuando se contraen simultáneamente (cuando se eleva la cabeza en posición de decúbito supino).

Laringe: