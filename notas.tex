ANATOMIA DE SUPERFICIE

CABEZA

Cara:

Huesos de la cara: la frente esta formada por la convexidad lisa del hueso frontal. Sus bordes inferiores (arcos supraciliares) dan lugar al borde superior de cada órbita.Estos dos arcos se unen con un puente denominado gabela. La depresión que hay debajo (entre la frente y la nariz) es el nasión.
El hueso maxilar articula con el frontal formando la cara interna de la órbita y junto con el hueso lacrimal aloja al aparato lacrimal.
La nariz  está formada por los huesos nasales que articulan entre sí y con los huesos frontal (por arriba) y maxilar (por el lado). Las suturas internasales y frontonasales se cruzan en el nasión.Además la parte más promiinente de la nariz esta formada por el cartílago nasal que separa las fosos nasales con el septo nasal
El borde externo de la órbita está formado por los huesos frontal y cigomático. Este último forma también la prominencia de la mejilla y el arco cigomático (junto con la apófisis cigomática del hueso temporal)y junto con el hueso maxilar el borde inferior de la órbita.
El borde inferior del maxilar presenta los alveolos para los dientes superiores, así como el hueso de la mandíbula contiene los de los huesos inferiores.

Músculos de la cara: Los músculos de la cara contribuyen mayoritariamente a dotar a la cara de expresión facial. Estos músculos se disponen como esfínteres (músculo orbicular del ojo y músculo orbicular de la boca) y dilatadores alrededor de la órbita, nariz y boca. Además existe el músculo frontal que se une con la apófisis occipital y la platisma que se une al tejido subcutáneo del cuello y del toráx superior.

El resto de la cabeza está formada por la bóveda craneal. La cara lateral de la bóveda la componen los huesos frontal parietal occipital y temporal.
Las apófisis de los huesos frontal y cigomático (que forman el borde externo de la órbita)forman el borde anterior de la fosa temporal. La línea temporal da inserción a la fascia del músculo temporal  (puede sentirse al masticar)que inferiormente se inserta en el arco cigomático.
El múculo masetero va desde la cara externa posteroinferior de la mandíbula al borde inferior del arco cigomático (se acentúa al apretar los dientes).
Detrás de la oreja se encuentra la apófisis mastoides (recubierta por el esternocleidomasoideo y el músculo digástrico).

Mandíbula: cuerpo, rama y ángulo. El cóndilo mandibular se palpa con dificultad con la mandíbula cerrada pues está cubierto por el ligamento lateral, pero se hace visible al abrir la mandíbula.

Órbita-globo ocular: limitada anteriormente por los párpados superior e inferior que estań unidos en sus extremos medial y lateral, limitando la hendidura palpebral. En sus bordes se encuentran las pestañas dispuestas en dos o tres hileras irregulares. En el lado medial se encuentra la carúncula lacrimal, que se une al saco lacrimal mediante los canalículos lacrimales (no visibles). Los párpados estan rodeados por el músculo orbicular y el superior se eleva por la acción del músculo elevador del párpado. Los movimientos oculares se deben a los músculos extraoculares.

Cavidad oral: Es el principio del tubo digestivo y su membrana mucosa comienza en la cara anterior de los labios (parte rosada). Contiene lengua, las arcadas dentarias con encías y dientesy recibe los orificios de salida de las glándulas salivares. Existen el paladar duro, formado por los huesos maxilar y palatino y el paladar blando que termina en la úvula. La lengua descanas sobre el suelo de la boca. El frenillo une la cara inferior de la lengua con el suelo de la boca.

Oreja: formada por piezas de fibrocartilago dispuestas de forma irregular a las que se une la piel. Hélix, antihélix, trago, antitrago, lóbulo, orificio auditivo externo.


CUELLO:

Visión anterior: Delimitado superiomente por la mandíbula (cuerpo y ángulos), inferiormente por la hendidura esternal del manubrio y las clavículas. Las clavículas articulan con el esternón (unión esternoclavicular) y con el acromion (unión acromioclavicular).
El músculo esternocleidomastoideo se inserta en la apófisis mastoides y tercio externo de la línea nucal superior por su límite superior y en el esternón y la clavícula por su parte inferior. Los bordes anteriores de la parte esternal y de la parte clavicular forman una V prominente cuando se contraen simultáneamente (cuando se eleva la cabeza en posición de decúbito supino).

Laringe: Se sitúa en la línea media del cuello. Puede verse cómo se eleva durante la deglución. En el borde superior se encuentra el hueso hioides (forma de U) a nivel de la cuarta vértebra cervical. El cartílago tiroides forma una prominencia en la línea media que es más evidente en el varón (angulo entre sus alas 90º, mientras en la mujer 120º). Las cuerdas vocales se insertan en su parte posterior y los músculos se unen al hueso con el hioides superiormente y al dorso del manubrio inferiormente. El cricoides forma parte del borde inferior de la laringe y es el único anillo completo de cartílago en la vía respiratoria (los demás están incompletos por su cara posterior). La glándula tiroides está formada por varios lóbulos y un itsmo que los une (línea media anterior al segundo o tercer cartílagos traqueales). La glándula no se aprecia, sin embargo si aumenta de tamaño (bocio) se puede ver cómo se mueve con la laringe. 

Región submandibular: El triángulo anterior (esternocleidomastoideo, rama mandibular y línea media) contiene otros tres triángulos formados por dos músculos insertados en el hueso hioides. El triángulo digástrico formado por el vientre anterior y posterior del músculo digástrico (unidos por un tendón al hueso hioides) y el borde inferior de la mandíbula. El triángulo carotídeo formado por el vientre posterior del músculo digástrico, el músculo omohioideo y el esternocleidomastoideo. Y el triángulo muscular delimitado por el omohioideo, el vientre anterior del digástrico y la línea media. El triángulo digástrico contiene la glándula submandibular que descansa sobre y alrededor del borde posterior del músculo milohioideo (de la lengua al hioides). 

Visión lateral:Delimitado superiormente por cuerpo, ángulo y ramas de la mandíbula y la articulación temporomandibular Inferiormente está delimitada por la clavícula y el acromion que forman el vértice del hombro (articulación acromioclavicular).
El triángulo posterior formado por esternocleidomastoideo, clavícula y músculo trapecio. En un individuo delgado con la cabeza vuelta y flexionada al lado contralateral se pueden ver el omohioideo cruzando el triángulo y el músculo escaleno anterior descendiendo.

Visión posterior: Delimitada superiormente por la línea nucal superior del hueso occipital que termina medialmente en la protuberancia occipital externa y lateralmente en la apófisis mastoides.
El ligamento nucal se encuentra en la línea media y se inserta en el hueso occipital y en las apófisis espinosas de las vértebras cervicales. Son visibles las apófisis espinosas de las dos o tres vertebras cervicales inferiores, así como todas las torácicas. La apófisis espinosa de la séptima vertebra cervical  y la primera torácica son las más evidentes. También son visibles, el acromion y la espina escapular.
El músculo trapecio se inserta superiormente en el tercio medio de la línea nucal superior.


TÓRAX

Pared torácica anterior: Desde las clavículas hasta el borde costal inferior. Formado por el esternón, las costillas y cartílagos costales y los músculos relacionados.
El esternón está formado por tres partes: el manubrio, el cuerpo y la apófisis xifoides. Se puede observar la escotadura external sobre la cara superior del manubrio. El manubrio y el cuerpo esternal son palpables y están unidos por una articulación cartilaginosa. Las costillas (visibles en individuos delgados) se cuentan desde esta localización: el cartilago de la segunda costilla se articula a cada lado de la articulación manubrioesternal. La apófisis xifoides se encuentra cubierta por el recto anterior del abdomen, por lo que no se aprecia superficialmente, cuando crece y se aprecia se debe considerar anormal.
La caja costal está formada por doce pares de costillas (parte ósea posterolateral y parte anterior catilaginosa). La primera costilla no de palpa fácilmente, ya que está por debajo del pectoral mayor y de la clavícula. Las siete primeras costillas (verdaderas) articulan directamente co el esternón, de la octava a la décima (falsas) Articulan con la costilla superior y las costillas once y doce son las flotantes. Las seis últimas costillas forman el reborde costal inferior. Los espacios intercostales están ocupados por los músculos intercostales, insertados en las costillas adyacentes.
La clavícula se observa en toda su longitud, en su cara interna articula con  el esternón (articulación esternoclavicular).
El músculo pectoral mayos se inserta en el borde interno de la clavícula, en los cartílagos de las cinco costillas superiores y en el húmero, formando el pliegue axilar anterior. El pectoral menor está cubierto por el mayor por lo que no se percibe superficialmente.
El músculo deltoides forma el contorno redondeado del hombro que recubre la articulaciom del hombro, insertándose en la clavícula y la escápula.
Los músculos de la pared abdominal se insertan en las costillas inferiores y en los cartílagos costales.
Todas las apófisis espinosas de las vértebras torácicas se palapan en la línea media. La espina y acromion de la escápula son visibles durante los movimientos del brazo. La escápula externamente se articula con la clavícula y el húmero, pero carece de inserción ósea medial. La moyoría de la caja costal no es visible desde la cara posterior, estando cubierta por los músculos erectores de la columna y la escápula y sus inserciones musculares.

Pared torácica lateral: Limitada inferiormente por el borde inferior de la caja costal y superiormente por la axila.
El músculo serrato anterior se inserta en el borde anterior interno de la escápula y en las ocho costillas superiores, formando ocho facículos que pueden verse en una persona delgada junto con sus interdigitaciones con el músculo oblicuo externo en las cuatro costillas medias.
La glándula mamaria femenina se extiende de la segunda costilla a la séptima y desde el borde externo del esternón hasta la pared axilar anterior. Descansa sobre los músculos  pectoral mayor, serrato anterior y oblicuo externo. El tamaño de la mama y la altura del pezón son variables en cada mujer. En el varón el pezón se encuentra en el cuarto espacio intercostal.
Axila: Es una cavidad rellena de grasa que separa el brazo del tórax. Su pared interna está formada por las seis costillas superiores y las paredes anterior(pectoral mayor) y posterior (músculo redondo mayor) convergen lateralmente sobre la porción superior de la diáfisis humeral.


ABDOMEN Y PELVIS

Pared abdominal anterior: Delimitada superiormente por el reborde costal inferior e inferiormente por la sínfisis del pubis, la tuberosidad púbica, el ligamento inguinal, la espina ilíaca anterosuperior y la cresta ilíaca (de medial a lateral). Los músculos abdominales forman una vaina fibrosa que contiene los músculos a cada lado de la línea media, las vainas de los dos lados se unen en la línea media formando un rafe fibroso, conocido como línea alba. Pueden apreciarse tres líneas transversales por encima del ombligo debidas a la inserción del recto anterior de abdomen a la vaina anterior, denominadas inserciones tendinosas (en un sujeto musculado). También se puede apreciar la línea semilunar que marca el borde lateral del recto. El músculo mas superficial de los músculos laminares del abdomen es el oblicuo externo que se inserta superiormente en las ocho costillas inferiores y tiene interdigitaciones con el serrato anterior en las cuatro costillas medias. Inferiormente se inserta en la sínfisis del pubis, tuberosidad púbica, espina ilíaca anterosuperior y mitad anterior de la cresta ilíaca y entre la espina y la tuberosidad púbica se vuelve sobre sí mismo formando el ligamento inguinal.

Región posteroinferior del tronco: Se encuentran las costillas inferiores, la cresta iĺíaca, las espinas ilíacas posteriores, superiores e inferiores, las apófisis espinosas lumbares y la cara posterior del sacro.
Curvas de la columna y movimientos del tronco: Las curvas y los discos intervertebrales confieren cierta elasticidad a la columna. La mayoría de los músculos de la espalda mantienen la posición erecta del cuerpo. Estando de pie, en reposo, el centro de gravedad se encuentra en la segunda vértebra del sacro. Los movimientos del cuerpo desplazan el centro de gravedad hacia delante y se requiere una gran masa de músculos posteriores para volver a la posición erecta. Flexión del tronco: Recto abdominal y músculos paravertebrales, extensión: músculos largos de la espalda, flexión lateral: músculos oblicuos y cuadrado lumbar, rotación:oblicuo interno y externo. Alteraciones en la columna vertebral pueden dar lugar a escoliosis o cifosis, que a su vez pueden coexistir (cifoescoliosis).

Región inginal, periné, escroto y pene: El músculo oblicuo externo se enrolla sobre sí mismo formando el ligamento inginal (va desde la espina ilíaca anterosuperior hasta la tuberosidad púbica). En el varón forma un orificio a través de la pared abdominal por el que el testículo pasa descendente para llegar al escroto. Los testículos pueden palparse dentro del escroto, al igual que los polos superior e inferior de l epidídimo. El cuerpo del pene termina ensanchándose en el glande del pene, que contiene un orificio externo llamado uretra. Recubriendo el glande se encuentra la piel del prepucio que se extirpa en la circuncisión.

Periné femenino: La apertura de la vagina esta rodeada por los labios mayores y los menores. Los labios mayores se unen anterosuperiormente alrededor del clítoris. La uretra desemboca entre el clítoris y la vagina. El ano esta situado en la linea media, alineado con las tuberosidades isquiáticas, anterior al coxis.


MIEMBRO SUPERIOR

Visión anterior del hombro y del brazo: la cintura escapular une el hueso húmero con el tronco. Está formada por la escápula y la clavícula y músculos que la sujetan. La clavícula articula medialmente con el esternón y la primera costilla (unión esternoclavicular).Lateralmente articula con la escápula (unión acromioclavicular).

El deltoides da forma al contorno del hombro cubriendo la epífisis proximal del húmero, insertándose en la clavícula, el acromion y la espina de la escápula. Lateralmente se inserta en el tubérculo deltoideo de la cara lateral del húmero. Es el principal abductor del brazo.

El músculo pectoral mayor tiene dos fascículos mediales. El clavicular que inserta en los dos tercios mediales de la clavícula. El esternocostal inserta en el esternón y en los seis cartílagos costales superiores. Lateralmente ambos fascículos convergen en un tendón  que se inserta por debajo del deltoides en el labio lateral del surco bicipital del húmero. Juntos forman la mayor parte de la cara anterior de la axila. Es un potente adductor. Junto al dorsal ancho hace prominencia al presionar las manos sobre las caderas.

La apófisis coracoides da inserción a los músculos coracobraquial, cabeza corta del bíceps y pectoral menor. Puede palparse un centímetro por debajo de la clavícula bajo las fibras mediales del deltoides.

El bíceps forma una prominencia en la parte anterior del brazo. Su cabeza corta inserta en la apófisis coracoides y la larga en la cara superior de la fosa glenoidea de la escápula. Distalmente se inserta en la tuberosidad bicipital del radio. Flexiona el codo y es también un supinador del antebrazo.

Visión posterior del hombro y del brazo: El acromion y la espina de la escápula son visibles.

El trapecio forma una ancha lámina triangular que se inserta medialmente en la mitad medial de la línea nucal superior del hueso occipital, el ligamento nucal, las apófisis espinosas y los ligamentos interespinosos de las vértebras cervicales inferiores y de todas las torácicas. Lateralmente se inserta en el tercio externo de la clavícula, el acromion y la espina de la escápula. Se encarga de un amplio rango de movimientos escapulares.

El músculo dorsal ancho también tiene una inserción medial ancha: apófisis espinosas lumbares,fascia lumbar y mitad posterior de la cresta ilíaca. También se inserta en el ángulo inferior de la escápula y ayuda a impedir que este ángulo sobresalga de la pared torácica durante los movimientos del hombro. Su tendón estrecho envuelve al músculo redondo mayor en la pared posterior de la axila. Es un potente adductor del brazo.

La parte inferior del pliegue axilar está formada principalmente por el redondo mayor, que va desde el borde lateral de la escápula hasta el surco bicipital del húmero. El borde lateral de la escápula puede percibirse a través de la masa del redondo mayor.

Los músculos infraespinoso (cara posterior de de la escápula por debajo de la espina) y subescapular (cara anterior de la escápula) forman el pliegue axilar superior. El músculo supraespinoso se inserta en la cara posterior de la escápula sobre la espina, pasa inferior al acromionhacia la tuberosidad mayor del húmero. Es importante para iniciar la abducción. Estos músculos junto con el redondo mayor refuerzan la articulación del hombro anterior, superior y posterior.

Movimientos de la escápula y de la articulación del hombro: los desplazamientos del miembro superior se producen gracias a los movimientos de la escápula y la clavícula.

La escápula se eleva al encogerse los hombros, mediante las fibras superiores del trapecio y el músculo elevador de la escápula. Forma parte del relieve muscular de la nuca en visión posterolateral. La escápula desciende por acción de la gravedad y del serrato anterior y el pectoral menor (estos dos también lo mueven hacia delante). La retracción de la escápula se realiza mediante el el trapecio y el romboides (profundo). La rotación externa de la escápula la realizan el serrato anterior y el trapecio y la rotación interna los músculos elevador de la escápula, pectoral menor y romboides. La flexión del hombro la realizan el facículo clavicular del pectoral mayor, las fibras anteriores del deltoides y el coracobraquial, mientras que la extensión la realizan las fibras posteriores del deltoides y, si se pate de la flexión, también el pectoral mayor y el dorsal ancho. La abducción la inicia el supraespinoso y la continúa el deltoides. La rotación interna: pectoral mayor, deltoides, y dorsal ancho. Rotación externa: deltoides, redondo menor e infraespinoso.