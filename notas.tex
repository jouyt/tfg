ANATOMIA DE SUPERFICIE

CABEZA

Cara:

Huesos de la cara: la frente esta formada por la convexidad lisa del hueso frontal. Sus bordes inferiores (arcos supraciliares) dan lugar al borde superior de cada órbita.Estos dos arcos se unen con un puente denominado gabela. La depresión que hay debajo (entre la frente y la nariz) es el nasión.
El hueso maxilar articula con el frontal formando la cara interna de la órbita y junto con el hueso lacrimal aloja al aparato lacrimal.
La nariz  está formada por los huesos nasales que articulan entre sí y con los huesos frontal (por arriba) y maxilar (por el lado). Las suturas internasales y frontonasales se cruzan en el nasión.Además la parte más promiinente de la nariz esta formada por el cartílago nasal que separa las fosos nasales con el septo nasal
El borde externo de la órbita está formado por los huesos frontal y cigomático. Este último forma también la prominencia de la mejilla y el arco cigomático (junto con la apófisis cigomática del hueso temporal)y junto con el hueso maxilar el borde inferior de la órbita.
El borde inferior del maxilar presenta los alveolos para los dientes superiores, así como el hueso de la mandíbula contiene los de los huesos inferiores.

Músculos de la cara: Los músculos de la cara contribuyen mayoritariamente a dotar a la cara de expresión facial. Estos músculos se disponen como esfínteres (músculo orbicular del ojo y músculo orbicular de la boca) y dilatadores alrededor de la órbita, nariz y boca. Además existe el músculo frontal que se une con la apófisis occipital y la platisma que se une al tejido subcutáneo del cuello y del toráx superior.

El resto de la cabeza está formada por la bóveda craneal. La cara lateral de la bóveda la componen los huesos frontal parietal occipital y temporal.
Las apófisis de los huesos frontal y cigomático (que forman el borde externo de la órbita)forman el borde anterior de la fosa temporal. La línea temporal da inserción a la fascia del músculo temporal  (puede sentirse al masticar)que inferiormente se inserta en el arco cigomático.
El múculo masetero va desde la cara externa posteroinferior de la mandíbula al borde inferior del arco cigomático (se acentúa al apretar los dientes).
Detrás de la oreja se encuentra la apófisis mastoides (recubierta por el esternocleidomasoideo y el músculo digástrico).

Mandíbula:cuerpo, rama y ángulo. El cóndilo mandibular se palpa con dificultad con la mandíbula cerrada pues está cubierto por el ligamento lateral, pero se hace visible al abrir la mandíbula.