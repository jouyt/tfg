\section{Conclusión}
Tras haber realizado el análisis de varias obras de distintas épocas se pueden sacar varias ideas en claro.

Las obras se rigen por el estilo predominante de la época, aunque en ocasiones los autores de éstas aporten novedades en el arte de su tiempo, como Mantegna que revolucionó la perspectiva con su lamentación de Cristo crucificado o Dalí con su estilo único.

Aunque todos los autores intentan mostrar una obra bella, algunos utilizan para ello proporciones y armonía, como Velázquez, y en otros prevalece el realismo.

En relación a la anatomía, se ve gran diferencia entre aquellas obras anteriores a las disecciones del cuerpo y aquellas en las que el conocimiento del cuerpo era mayor. La figura de la obra de Mantegna no posee una anatomía muy estudiada, Velázquez se centra mucho en la anatomía superficial y las proporciones del Cristo de su obra son adecuadas, y Dalí se propone dibujar el Cristo más bello, para el que adopta una perspectiva nueva e incluso se dice que utilizó un modelo para representar de forma correcta la anatomía.

En cuanto a la representación de la muerte de Cristo se puede decir que el que mejor la representa es Mantegna, puesto que la figura de su obra presenta laxitud en la postura y rigidez en algunos músculos, que probablemente está relacionada con el \textit{rigor mortis}, además de la lividez con la que le ilustra el autor, que reproduce de manera bastante acertada la apariencia de un cadáver.

El número de clavos con el que Cristo aparece clavado en la Cruz es variable debido, más a las ideas del propio autor que a las de la época. De hecho, en todas las obras analizadas se puede observar la inserción de tres clavos, uno uniendo los dos pies y otro en cada extremidad superior, excepto en la obra de Velázquez que se deja influir por su maestro, y no así por su tiempo. Otro detalle sobre la inserción de los clavos es que a lo largo del tiempo el Evangelio de San Juan ha sido interpretado de forma que incluso en épocas más modernas estos se han seguido representado en la palma de la mano.