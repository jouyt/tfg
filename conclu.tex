\section{Conclusión}
Tras haber realizado el análisis de varias obras de distintas épocas se pueden sacar varias ideas en claro.

Las obras se rigen por el estilo predominante de la época, aunque en ocasiones los autores de éstas aporten novedades en el arte de su tiempo, como Mantegna que revolucionó la perspectiva con su lamentación de Cristo crucificado o Dalí con su estilo único.

La intención del autor también cambia. Aunque se intente mostrar una obra bella, algunos autores utilizan para ello proporciones y armonía, como Velázquez o Dalí, mientras que otros utilizan la prevalencia del realismo para crear una obra que represente la muerte de Cristo de una forma lo más exacta posible a la realidad, éste es el caso de Mantegna o Miñarro.

En relación a la anatomía, %se ve diferencia entre aquellas obras anteriores a las disecciones del cuerpo y aquellas en las que el conocimiento del cuerpo era mayor.
la figura de la obra de Mantegna posee una anatomía estudiada, sin embargo no es en lo que más se centra, numerosos detalles en la obra son más impresionantes, como la perspectiva que adopta la figura de Cristo o los rasgos referentes a la muerte que éste presenta. Velázquez sí que se centra mucho en la anatomía superficial y en las proporciones del Cristo de su obra que son adecuadas. Dalí, por su parte, se propone dibujar el Cristo más bello, para el que adopta una perspectiva nueva e incluso se dice que utilizó un modelo para representar de forma correcta la anatomía. Miñarro, por su parte, utiliza la Síndone de Turín, para intentar representar la muerte de Cristo como nunca hasta ahora se había visto, en toda su crudeza.

El número de clavos con el que Cristo aparece clavado en la Cruz es variable debido, más a las ideas del propio autor que a las de la época. De hecho, en todas las obras analizadas se puede observar la inserción de tres clavos, uno uniendo los dos pies y otro en cada extremidad superior, excepto en la obra de Velázquez que se deja influir por su maestro, y no así por su tiempo. Otro detalle sobre la inserción de los clavos es que a lo largo del tiempo el Evangelio de San Juan ha sido interpretado de forma que incluso en épocas más modernas estos se han seguido representado en la palma de la mano y no ha sido hasta hace varias décadas que se han hecho experimentos en cuanto a ello. Por ello, en la obra de Miñarro es en la única que se pueden apreciar los clavos a la altura de los carpos y no entre los metacarpos como se representa en las demás obras.

En cuanto a la representación de la muerte de Cristo, Mantegna en su obra presenta la laxitud en la postura y la rigidez en algunos músculos, que probablemente está relacionada con el \textit{rigor mortis}, además de la lividez con la que el autor ilustra el cuerpo, reproduciendo de manera bastante acertada la apariencia de un cadáver.

Velázquez, sin embargo, realiza una obra en la que no reproduce la muerte de Cristo. El Cristo no posee ninguna característica de un cuerpo muerto, ni laxitud en los músculos ni lividez cadavérica. Es más, muestra ciertas características que nos llevan a la conclusión de que el Cristo no ha fallecido e icluso se encuentra en relativas buenas condiciones, sobre todo, tratándose de una figura que ha soportado toda clase de martirios.

Dalí propone la imagen de un Cristo que no porta nungún elemento utilizado para su suplicio ni crucifixión. Aún así, nos muestra una cristo cuyos músculos están relajados, como si estuviese a punto de fallecer o ya hubiera fallecido. En cuanto a la coloración del cuerpo, es dificil de determinar si posee el característico del \textit{livor mortis} o no, debido al gran contraste de luz que Dalí utiliza en la obra.

En el Cristo de la Síndone de Miñarro se distinguen fácilmente las características de la muerte en la figura. De hecho ésta posee un \textit{rigor mortis} con una rigidez muy marcada y un \textit{livor mortis} que se aprecia en el color ceniciento del cuerpo, así como en el acúmulo de sangre en la zona más distal de las extremidades inferiores.