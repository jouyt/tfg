\section{Análisis de la obra pictórica: Lamentación sobre Cristo muerto de Andea Mantegna}

Arte, estilo: Renacimiento italiano, quattrocento

Cronología: 1480

Lugar: Pinacoteca de Brera, Milán

Autor: Andrea Mantegna

Título: Lamentación sobre Cristo muerto

Función:

Contexto histórico: Andrea Mantegna realiza esta obra en pleno período humanista, el centro del mundo ya no es Dios o la religión, sino que el hombre pasa a ser el centro. Esto derivaría en ideas de libertad, rivalidad, competencia, que no hacen otra cosa que contribuir al gran avance que se da en el arte durante el Renacimiento en general.

Es durante esta época que los pintores comienzan a tomar importancia como artistas, no sólamente como meros técnicos. A partir de este momento empiezan a estudiar formas de hacer las obras más cercanas y reales para el espectador, se busca la consecución de una imagen con volumen, profundidad y dimensiones adecuadas. De esta manera se dan verdaderos progresos en la perspectiva o en las proporciones.

La obra analizada es, de hecho, una obra pionera en la perpectiva con la que representa a Cristo muerto. En épocas anteriores y en otras obras de la misma época en las que se representa a Cristo muerto, lo usual era contemplarlo de perfil. En la búsqueda de realismo en su obra el autor ladea la cabeza y los pies de la figura, eliminando, de esta forma, la estaticidad que le aporta la simetría del cuerpo, la cual también se ve anulada mediante las arrugas de tela que cubren la pelvis y las piernas de Cristo y las formas laxas del cuerpo.

Se trata de una obra dramática del cuerpo de Cristo muerto, antes de la resurrección y el triunfo sobre la muerte. Este dramatismo se percibe en la expresión tanto de la cara de la figura principal, como los rostros de aquellos que contemplan con horror la escena y se incrementa con los colores utilizados, de la misma gama cromática, la austera decoración del resto de la estancia, en la que solo vemos la lápida en la que se encuentra tumbado Cristo y la almohada en la que apoya la cabeza.

%http://www.arteespana.com/quattrocentoitaliano.htm
%http://www.arteespana.com/andreamantegna.htm
%http://www.artehistoria.jcyl.es/v2/obras/4376.htm
%http://educacion.ufm.edu/andrea-mantegna-lamentacion-sobre-cristo-muerto-oleo-sobre-tela-en-torno-a-1480-1490/
%http://cv.uoc.edu/~04_999_01_u07/percepcions/perc57.html
%Libro de Olatz

\vspace{12pt}
\textbf{Anatomía de superficie:}

Andrea Mantegna sorprende en su época con esta obra en la que Cristo se presenta en una perspectiva nunca antes vista, y con una figura de proporciones y anatomía exquisitas. Debido precisamente a la perspectiva en la que se encuentra la figura no es fácil discernir el modelo proporcional que sigue, aunque si es apreciable la armonía en la proporción de la figura, con una cabeza adecuada al tamaño del resto del cuerpo.

En esta obra se aprecia de manera sublime la muerte de Cristo. La laxitud de los músculos de las extremidades y la lividez que presenta el cuerpo nos da una idea bastante acertada de la imagen de un cadáver. La palidez amarillenta con la que el autor refleja el cuerpo y los tonos grisáceos incluso amoratados hacen visible lo que se denomina el \textit{pallor mortis}. Esta es una de las fases que acontece a nuestro cuerpo una vez el corazón deja de bombear sangre y, por tanto, esta no llega a los capilares, produciendo una palidez cadavérica. Una vez la sangre deja de circular tiende a por gravedad a dirigirse a la parte baja del cuerpo, que en el caso de la figura de la obra sería en su zona dorsal. En la obra se aprecian zonas más oscuras que podrían coincidir con estas zonas amoratadas del \textit{livor mortis}, pero podrían, de igual manera, tratarse de la propia sombra que proyecta la figura.

Se contemplan los orificios creados por los clavos en primer plano, tanto en los pies como en las manos, representados con gran realismo por el autor. Se pueden observar estos según las creencias de la época en las palmas de las manos, en lugar de a la altura de las muñecas, donde de acuerdo con la creencia actual, eran introducidos.
Las heridas, tanto las originadas por los clavos como la producida por la lanza, se encuentran limpias y sin una gota de sangre. Tanto es así que la del costado derecho apenas es visible en la perspectiva de la figura.

Se puede apreciar sin esfuerzo el volumen de las distintas partes del cuerpo, que el autor plasma duramente, casi a modo escultórico. 

Se percibe la laxitud de todos los músculos del cuerpo muerto, pudiendo, además, observar estructuras que en obras tan antiguas no solían aparecer, como son las plantas de los pies, donde apreciamos diversas estructuras anatómicas: el arco plantar, caramente definido y el calcáneo y las cabezas de los metatasianos, que están cubiertos por grasa subcutánea que sirve de almohadillamiento.

También se perfila de forma aparente la caja torácico, tras un vientre hundido, que hace notable el borde costal inferior. Y anunque las estructuras no están claramente precisadas se puede intuir, debido al vientre hundido, la espina ilíaca en ambos costados de la pelvis.

Las extremidades superiores caen inertes a ambos lados de la figura, con cierta flexión en las articulaciones del codo, muñeca, metacarpofalángica e interfalángicas probablemente debido al \textit{rigor mortis}. Éste se define como la rigidez que adquieren los musculos tras varias horas de la muerte debido a cambios químicos en las células del cuerpo. Refiriéndome a este mismo proceso, he de mencionar la rigidez de las piernas, puesto que si tal caso no se diera los pies caerían laxos hacia los lados.



La espalda se encuentra erguida, los músculos que se encargan de mantener esta postura son los principales erectores de la columna: los músculos del grupo iliocostal, los del grupo longuísimo y los del espinoso.

El músculo dorsal ancho, que además se puede apreciar en la obra, proviene de la espalda e inserta en el ángulo inferior de la escápula. Su tendón estrecho envuelve al músculo redondo mayor que forma el pliegue posterior de la axila. Ambos músculos junto con el pectoral mayor, el que además forma el pliegue anterior de la axila, elevan el tronco cuando los brazos se encuentran fijos, como es el caso de la figura examinada.

El músculo trapecio, cuya principal función es la elevación de la escápula, también se observa. El serrato anterior gira la escápula y, por tanto, junto con el músculo trapecio y el elevador de la escápula que la elevan hacen que la cavidad glenoidea se oriente hacia arriba y adelante, como en la figura de la obra de Velázquez.
El deltoides, que se ve fácilmente redondeando y sustentando la articulación del hombro por su parte superior, es el principal abductor del brazo, ya que sigue el movimiento abductor que el músculo supraespinoso inicia, por lo que tiene gran relevancia en la posición de crucifixión en la que ambos brazos se encuentran en posición de abducción.

En el brazo se observan tanto el bíceps como el tríceps. El bíceps forma una prominencia en la parte anterior del brazo y se encarga de la supinación del antebrazo, la cual realiza con la ayuda del músculo supinador. El tríceps por su parte se encarga de la extensión de la articulación del codo. En este movimiento le secunda el músculo ancóneo, que no es visible en la figura de la obra analizada por encontrarse en la parte posterior del codo.

La mano se encuentra en posición de reposo en la que las articulaciones falángicas y metacarpofalángicas se encuentran ligeramente flexionadas. El músculo palmar largo contribuye sutilmente a la flexión de las articulaciones metacarpofalángicas, que es realizada fundamentalmente por los músculos flexores de los dedos.

Al tratarse de un individuo delgado y favorecido por la postura en que se encuentra (brazos estirados hacia los lados y hacia arriba) se puede intuir la caja torácico perfectamente. El borde costal inferior formado por las seis últimas costillas es fácilmente visible, al igual que la depresión en la que se encuentra el cuerpo del esternón y el contorno de algunas costillas (probablemente la cuarta, quinta, sexta y séptima).

En el abdomen se encuentran los músculos abdominales, que se pueden apreciar, aunque no muy marcados. Estos músculos forman una vaina fibrosa a cada lado de la línea media. Ambas vainas se unen en esta, formando la línea alba. También se divisa bastante bien la línea semilunar, que marca el borde lateral del músculo recto del abdomen cuya principal función en la figura es el mantenimiento de la postura erecta apoyando así a los músculos erectores de la columna.

El músculo mas superficial de los músculos laminares del abdomen es el oblicuo externo. Este entre la espina y la tuberosidad púbica gira sobre sí mismo y forma el ligamento inguinal. La depresión que suele haber a la altura de ese ligamento, y que suele ser más visible en hombre, se puede observar perfectamente en la obra pictórica.

La posición de las piernas, en las que la rodilla derecha de la figura se encuentran una en posición de extensión, mientras que la otra se encuentra ligeramente flexionada, colaboran varios músculos. Los encargados de la extensión son: el cuadriceps y los músculos del tracto iliotibial, y los de la flexión son: los músculos poplíteos y los gemelos. Al estar de pie soportando el peso sobre una pierna, el lado de la pelvis que no soporta el peso se eleva. Esta acción está realizada por los glúteos medio y menor y deja a la vista la espina ilíaca del lado derecho.
La rótula o patella se puede apreciar, al igual que la tuberosidad tibial, sobre todo en la rodilla flexionada.

Los pies se encuentran apoyados en una tabla, soportando el peso del cuerpo, y se ubican en paralelo entre ellos ligeramente separados anteriormente, con un clavo en cada dorso y sangre que emana de la herida. En la parte anterior de éstos, los dedos se encuentran relajados.
