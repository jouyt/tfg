\section{Proporciones del cuerpo humano} %http://es.scribd.com/doc/30286487/Proporciones-Del-Cuerpo-Humano
A lo largo de la historia han ido cambiando las representaciones del cuerpo humano, ya sea por razones ideológicas, estéticas o de conocimiento anatómico.

El deseo de representar al ser humano perfecto impulsó el estudio de las proporciones del cuerpo humano. Esto hizo que se diesen numerosos intentos por representar al ser humano de proporciones perfectas, pudiendo observar este fenómeno desde la antigüedad, en Egipto y Mesopotamia, hasta el hombre de Vitruvio, fruto de los estudios en este ámbito de Leonardo Davinci, y pasando por numerosas representaciones que reflejan los cánones de distintas épocas (Grecia y Roma, Edad Media; y posteriormente, Edad Moderna y Contemporánea).

%Se define cánon como un patrón concreto que fija las proporciones ideales del cuerpo humano en cuanto a su representación pictórica, en este caso.

La mayoría de los artistas confeccionan su canon del cuerpo humano tomando como referencia la cabeza para elaborar las proporciones del resto del cuerpo.
Según la época y el artista los cánones han ido variando, ya que se adaptan al ideal de belleza de la época, que a su vez tiene su origen en los cambios sociales que acontecen en cada período. En el presente se considera como figura ideal el canon de ocho cabezas de altura. Realmente los seres humanos medimos más cerca de siete cabezas de altura que de ocho. Por ello existen tres cánones para determinar las proporciones de la figura humana: el canon de siete cabezas y media, el canon más realista en cuanto a medidas, que se considera la figura común, el canon de ocho cabezas considerado, como ya he dicho, la figura ideal y el canon de ocho cabezas y media, con el que se representan las figuras heroicas.

En la representación femenina es preciso considerar ciertas diferencias anatómicas como la longitud de la cabeza, menor que la del hombre, que hace que su figura sea más baja; los hombros, que son más estrechos; la cintura, más ceñida; las caderas, más anchas; o el ombligo, que está situado más abajo que el del hombre, al igual que el pecho.

Hay que tener en cuenta que los niños guardan otras proporciones. Aunque durante la infancia estamos en constante crecimiento y es difícil crear un canon exclusivo, sí existen algunos cánones aproximados: de cinco cabezas para niños de dos años, de seis cabezas entre seis y doce y de siete entre los doce y los quince.

%http://es.scribd.com/doc/58358634/Canon-de-La-Figura-Humana