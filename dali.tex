\section{Análisis de la obra pictórica: Cristo de San Juan de la Cruz de Dalí}

Arte, estilo: Surrealismo

Cronología: 1951

Lugar: Museo Kelvingrove, Glasgow, Reino Unido

Autor: Salvador Dalí

Título: Cristo de San Juan de la Cruz

Función: Se expuso por primera vez en la galería Lefevre de Londres. %http://www.xn--esarteespaol-jhb.es/contenido.php?recordID=11

Función:

Contexto histórico: Esta obra se realiza en plena dictadura franquista, en la que numerosos artistas fueron desterrados por no compartir las ideas de la dictadura. No siendo así para Dalí que no sólo acepta la dictadura franquista, sino que incluso elaboró un retrato de la nieta al propio Franco.

A partir del 1948, cuando regresa a Europa de América, que inicia una etapa en la que vuelve al clacisismo, reproduciendo sus primeros temas religiosos. Una de estas obras religiosas es la obra del Cristo de San Juan de la Cruz. Este Cristo está basado, como su nombre indica, en en el Cristo dibujado por San Juan de la Cruz tras una revelación.

Dalí decide representar \textit{un Cristo bello como el mismo Dios que él encarna} y no centrase en la fealdad para provocar la emoción en el espectador. Lo hace adoptando una perspectiva totalmente nueva de la crucifixión. Este nuevo enfoque situará a Cristo en la Cruz en una posición vertical y casi perpendicular al espectador de la obra. Además decide retirar todos aquellos elementos que intervienen en la crucifixión de Cristo, nótese que éste no conserva la corona de espinas ni ninguna herida, tampoco están dibujados los los clavos que lo deberían sostener en la cruz, en la que parece que está suspendido por arte de magia.

Debajo de la imagen principal del cuadro se puede apreciar un paisaje, supuestamente representado a partir de la bahía de Port Lligat. En ella se ven dos pescadores que, sin embargo, están inspirados en los pintores Le Nain y Velazquez.

El Cristo se sitúa en un fondo oscuro que le da una imagen dramática a la obra junto al contraste con la iluminación que se proyecta en forma de rayo de luz sobre la figura. Entre la imagen del crucificado y la de los pescadores se interpone un cielo nuboso, que aporta a la obra aún mayor dramatismo al separar la crucifixión y la imagen de la tranquila bahía de pescadores, dando a entender la indiferencia ante el suceso que se está dando.

% http://www.artehistoria.jcyl.es/v2/obras/9639.htm
% http://unapizcadecmha.blogspot.com.es/2013/11/el-cristo-de-san-juan-de-la-cruz-1951.html
% http://www.arteespana.com/salvadordali.htm
% Artículo de San Juan de la Cruz
% libro de aita

\vspace{12pt}
\textbf{Anatomía de superficie:}
