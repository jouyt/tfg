\section{Análisis de la obra pictórica: Cristo de San Juan de la Cruz de Dalí}

Arte, estilo: Surrealismo

Cronología: 1951

Lugar: Museo Kelvingrove, Glasgow, Reino Unido

Autor: Salvador Dalí

Título: Cristo de San Juan de la Cruz

Función: Se expuso por primera vez en la galería Lefevre de Londres. %http://www.xn--esarteespaol-jhb.es/contenido.php?recordID=11

Función: Contexto histórico: Esta obra se realiza en plena dictadura franquista, en la que numerosos artistas fueron desterrados por no compartir las ideas de la dictadura. No siendo así para Dalí que no sólo acepta la dictadura franquista, sino que incluso elaboró un retrato de la nieta al propio Franco.

Gran pintor de estilo surrealista, Dalí sigue un estilo propio y más extravagante que el del resto de sus compañeros. El surrealismo nace en 1924, en París a manos de André Breton y se extiende hasta el final de la segunda guerra mundial. Dalí enseguida sobresale en este estilo con una primera etápa más agresivo. Sin embargo, a partir del 1948, cuando regresa a Europa de América, inicia una etapa en la que vuelve al clacisismo, reproduciendo sus primeros temas religiosos. Una de estas obras religiosas es la obra del Cristo de San Juan de la Cruz. Este Cristo está basado, como su nombre indica, en el Cristo dibujado por San Juan de la Cruz tras una revelación.

Dalí decide representar \textit{un Cristo bello como el mismo Dios que él encarna} y no centrase en la fealdad para provocar la emoción en el espectador. Lo hace adoptando una perspectiva totalmente nueva de la crucifixión. Este nuevo enfoque situará a Cristo en la Cruz en una posición vertical y casi perpendicular al espectador de la obra. Además decide retirar todos aquellos elementos que intervienen en la crucifixión de Cristo, nótese que éste no conserva la corona de espinas ni ninguna herida, tampoco están dibujados los los clavos que lo deberían sostener en la cruz, en la que parece que está suspendido por arte de magia.

Debajo de la imagen principal del cuadro se puede apreciar un paisaje, supuestamente representado a partir de la bahía de Port Lligat. En ella se ven dos pescadores que, sin embargo, están inspirados en los pintores Le Nain y Velazquez.

El Cristo se sitúa en un fondo oscuro que le da una imagen dramática a la obra junto al contraste con la iluminación que se proyecta en forma de rayo de luz sobre la figura. Entre la imagen del crucificado y la de los pescadores se interpone un cielo nuboso, que aporta a la obra aún mayor dramatismo al separar la crucifixión y la imagen de la tranquila bahía de pescadores, dando a entender la indiferencia ante el suceso que se está dando.

% http://www.artehistoria.jcyl.es/v2/obras/9639.htm
% http://unapizcadecmha.blogspot.com.es/2013/11/el-cristo-de-san-juan-de-la-cruz-1951.html
% http://www.arteespana.com/salvadordali.htm
% Artículo de San Juan de la Cruz
% libro de aita

\vspace{12pt}
\textbf{Anatomía de superficie:}

En esta obra en la que Dalí refleja una nueva perspectiva de la crucifixión de Cristo, siendo la cabeza de éste el centro de la obra, podemos observar un Cristo de con características totalmente diferentes a las apreciadas en la mayoría de obras acerca de la crucifixión. Aparte de tener el pelo corto, al contrario que en las numerosas obras en las que se representa a Cristo, no posee signos de sufrimiento, ni clavos que le sujeten a la cruz, ni corona de espinas, ni tampoco sangre puesto que no hay heridas de las que pueda brotar.

Aún sin elementos de crucifixión la figura mantiene una posición en la que se sugiere que ambos pies estarían unidos a la cruz mediante el mismo clavo, mientras que los clavos de las extremidades superiores se encontrarían en las palmas de las manos. Esto último se puede percibir en la caída del cuerpo hacia delante que ocurre desde más arriba de las muñecas, según la sombra que se proyecta detrás.

Al estar la cabeza inclinada hacia delante, no es posible ver el resto del cuerpo de la figura, sin embargo si ofrece una visión espectacular de la espalda y de sus músculos. Los brazos caen extendidos, formando un triángulo con la cruz que mantiene suspendido a Crsito y dando una sensación de que éste se encuentra colgado boca abajo por el espacio que hay entre su cuerpo y la cruz.

El autor en esta obra de arte no se centra en el sufrimiento y la muerte de Cristo, sino que intenta evocar la belleza de Cristo Dios. Por ello se puede considerar admirable la simetría del cuerpo y la anatomía que se vislumbra. La actitud de la figura es relajada, pudiéndose observar este hecho en la posición de sus brazos y de los músculos de la espalda.

No es posible saber si el Cristo se encuentra fellecido o al borde de serlo en la obra, puesto que el autor ilumina la figura con una luz amarillenta que crea un gran contraste entre las zonas iluminadas y las zonas en sombra y no deja observar el verdadero color del cuerpo de la figura.

Diferentes estructuras anatómicas se pueden observar en esta obra de acuerdo a la perspectiva y posición de la figura, que no coinciden con las descritas en figuras anteriores.

Los músculos de la espalda, aunque totalmente relajados, son claramente visibles.



El músculo dorsal ancho, que además se puede apreciar en la obra, proviene de la espalda e inserta en el ángulo inferior de la escápula. Su tendón estrecho envuelve al músculo redondo mayor que forma el pliegue posterior de la axila. Ambos músculos junto con el pectoral mayor, el que además forma el pliegue anterior de la axila, elevan el tronco cuando los brazos se encuentran fijos, como es el caso de la figura examinada.

El músculo trapecio, cuya principal función es la elevación de la escápula, también se observa. El serrato anterior gira la escápula y, por tanto, junto con el músculo trapecio y el elevador de la escápula que la elevan hacen que la cavidad glenoidea se oriente hacia arriba y adelante, como en la figura de la obra de Velázquez.
El deltoides, que se ve fácilmente redondeando y sustentando la articulación del hombro por su parte superior, es el principal abductor del brazo, ya que sigue el movimiento abductor que el músculo supraespinoso inicia, por lo que tiene gran relevancia en la posición de crucifixión en la que ambos brazos se encuentran en posición de abducción.

En el brazo se observan tanto el bíceps como el tríceps. El bíceps forma una prominencia en la parte anterior del brazo y se encarga de la supinación del antebrazo, la cual realiza con la ayuda del músculo supinador. El tríceps por su parte se encarga de la extensión de la articulación del codo. En este movimiento le secunda el músculo ancóneo, que no es visible en la figura de la obra analizada por encontrarse en la parte posterior del codo.

La mano se encuentra en posición de reposo en la que las articulaciones falángicas y metacarpofalángicas se encuentran ligeramente flexionadas. El músculo palmar largo contribuye sutilmente a la flexión de las articulaciones metacarpofalángicas, que es realizada fundamentalmente por los músculos flexores de los dedos.

Al tratarse de un individuo delgado y favorecido por la postura en que se encuentra (brazos estirados hacia los lados y hacia arriba) se puede intuir la caja torácico perfectamente. El borde costal inferior formado por las seis últimas costillas es fácilmente visible, al igual que la depresión en la que se encuentra el cuerpo del esternón y el contorno de algunas costillas (probablemente la cuarta, quinta, sexta y séptima).

En el abdomen se encuentran los músculos abdominales, que se pueden apreciar, aunque no muy marcados. Estos músculos forman una vaina fibrosa a cada lado de la línea media. Ambas vainas se unen en esta, formando la línea alba. También se divisa bastante bien la línea semilunar, que marca el borde lateral del músculo recto del abdomen cuya principal función en la figura es el mantenimiento de la postura erecta apoyando así a los músculos erectores de la columna.

El músculo mas superficial de los músculos laminares del abdomen es el oblicuo externo. Este entre la espina y la tuberosidad púbica gira sobre sí mismo y forma el ligamento inguinal. La depresión que suele haber a la altura de ese ligamento, y que suele ser más visible en hombre, se puede observar perfectamente en la obra pictórica.

La posición de las piernas, en las que la rodilla derecha de la figura se encuentran una en posición de extensión, mientras que la otra se encuentra ligeramente flexionada, colaboran varios músculos. Los encargados de la extensión son: el cuadriceps y los músculos del tracto iliotibial, y los de la flexión son: los músculos poplíteos y los gemelos. Al estar de pie soportando el peso sobre una pierna, el lado de la pelvis que no soporta el peso se eleva. Esta acción está realizada por los glúteos medio y menor y deja a la vista la espina ilíaca del lado derecho.
La rótula o patella se puede apreciar, al igual que la tuberosidad tibial, sobre todo en la rodilla flexionada.

Los pies se encuentran apoyados en una tabla, soportando el peso del cuerpo, y se ubican en paralelo entre ellos ligeramente separados anteriormente, con un clavo en cada dorso y sangre que emana de la herida. En la parte anterior de éstos, los dedos se encuentran relajados.
