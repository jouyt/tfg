\section{Introducción} % Para los títulos
El arte podría definirse como una expresión de la creatividad humana que produce obras apreciadas por su belleza o emotividad y que es innata en el ser humano \footnote{Definición según el Diccionario Oxford: www.oxforddictionaries.com/definition/english/art}. Existe desde el origen del ser humano y  % (http://www.oxforddictionaries.com/definition/english/art)
a lo largo de los siglos ha ido evolucionando hasta el arte contemporáneo. Junto con las obras artísticas también se han ido transformando las representaciones anatómicas de obras pictóricas y escultóricas.

La anatomía de superficie es la ciencia que estudia las características anatómicas que pueden ser estudiadas mediante la vista y la palpación \footnote{Definición según la enciclopedia Wikiradiography: www.wikiradiography.com/page/Surface+Anatomy}% (http://www.wikiradiography.com/page/Surface+Anatomy)
, es decir, la superficie del cuerpo. En el caso de la anatomía humana estudia desde las proporciones del cuerpo hasta puntos de referencia visibles en el exterior que están relacionados con órganos y partes internas, tanto en pose estática como en movimiento. % (http://www.princeton.edu/~achaney/tmve/wiki100k/docs/Superficial\_anatomy.html)
% %http://www.theodora.com/anatomy/surface\_anatomy\_index.html

%Como la materia principal de mi trabajo es la antomía de superficie relataré lo más relevante acerca de cómo ha evolucionado el conocimiento de esta materia hasta llegar al conocimiento actual.
\subsection{Historia de la anatomía}
A lo largo de la historia ha habido diferentes impedimentos que han provocado que hasta épocas muy recientes la anatomía del cuerpo humano fuese una gran desconocida, incluso entre aquellos que fundamentalmente trabajaban vinculados al cuerpo humano por distintas razones, como pueden ser médicos o artistas.

Desde el antiguo Egipto se ha estudiado la anatomía del ser humano. Conocían algunas estructuras como el corazón, el hígado, el bazo, los riñones e incluso la existencia de vasos sanguíneos que partían del corazón y otros vasos que transportaban distintos fluidos.

Los griegos fueron los grandes anatómicos de la antigüedad, de hecho, se recogen bastantes textos médicos en el Corpus Hipocraticum \footnote{Colección de trabajos médicos elaborados en la antigua Grecia. A pesar del su nombre no está comprobado que Hipócrates sea el autor de alguno de estos escritos, de hecho se desconoce el autor de la mayoría de ellos.}. En estos se describen la función de algunos órganos, así como del sistema músculo-esquelético y también la diferencia entre arterias y venas. Fueron los primeros en formar escuelas de anatomía e incluso los primeros en diseccionar cuerpos humanos \footnote{Ptolomeo I fue el primero que aprueba la utilización de cuerpos humanos para su disección y estudio anatómico. Durante su reinado y el de su dinastía fue posible para algunos anatomistas la realización de disecciones. El más importante es Herophilos considerado como "Padre de la anatomía".}, lo que permitió un conocimiento más avanzado del cuerpo humano. Sin embargo, esto no era lo habitual y normalmente utilizaban sus conocimientos de las disecciones animales para aplicarlas a la anatomía humana.

Galeno siguió con los estudios anatómicos mediante la disección de animales aportando además otros conocimientos que recibía gracias a su condición de médico entre los gladiadores donde podía vislumbrar cualquier tipo de herida. Sus estudios se encuentran recogidos en sus numerosos tratados. \footnote{Galeno con más de 500 tratados fue el autor de la antigüedad que más escritos elaboró}

El conocimiento anatómico del ser humano apenas avanza durante la Edad Media, debido a la creencia cristiana de que el alma de un cuerpo diseccionado no podría ir al ``Cielo". Por ello las disecciones son un tema tabú durante esta época.

Posteriormente, ya en la Edad moderna, se conceden permisos para diseccionar cadáveres de criminales como una parte más de la pena por su delito.

No es hasta Leonardo Da Vinci que se empiezan a hacer verdaderos conocimientos de la anatomía humana. Sus conocimientos se debieron a disecciones que llevó a cabo en cuerpos humanos y están plasmadas en una gran variedad de dibujos \footnote{El Hombre de Vitruvio es la más conocida de todas las ilustraciones elaboradas por Da Vinci.}. Aunque el verdadero cambio de la anatomía galénica a una doctrina más moderna se da en el siglo XVI gracias a Andrés Vesalio. Éste es el autor de "De Humani Corpis Fabrica", donde se recogen diversos dibujos de anatomía humana. A partir de este momento los estudiantes de anatomía no tenían que saber latín para poder estudiar y lo podían hacer mediante las diversas representaciones que se estaban llevando a cabo.

Las disecciones empiezan a ser más comunes, y se crean ahora varios problemas.

El primer problema se debe a la temperatura del ambiente. En aquella época no se conocía la manera de conservar un cuerpo. Es por esto que la disseciones se podían realizar exclusivamente en los meses de temperatura más baja, pues el frío mantiene el cadáver por más tiempo. Aún así había que realizar la disección casi inmediatamente después de la muerte del individuo, pues los cuerpos no tardan mucho en descomponerse.

El segundo problema proviene del aumento de las disecciones. Debido a los pocos cuerpos que permitían diseccionar y al aumento de estudiantes de anatomía se empezaron a robar tumbas para conseguir cadáveres, algunos incluso llegaron más lejos matando a personas para vender sus cuerpos \footnote{Conocidos como los asesinatos de West Port o los asesinatos de Burke y Hare, fueron llevados a cabo durante 1828 en Edimburgo. William Burke y William Hare asesinaron a dieciséis personas y vendieron sus cuerpos al Doctor Robert Knox, quien los utilizaba en sus clases de anatomía. Cuando se descubrieron los crímenes Hare testificó contra Burke quien fue condenado a muerte y después diseccionado.}.

Para evitar este último problema se elaboró la ley de 1832 que regulaba las donaciones de cuerpos a la ciencia y la obtención de licencias para obtener el permiso que permitía diseccionar cadáveres.

Durante los siglos XVII y XVIII el campo de la anatomía se desarrolla tremendamente. Al principio se realizaban disecciones en plazas donde cualquiera podía atender a la explicación de un maestro, después, no obstante las disecciones pasan a realizarse en aulas siendo muchos menos los beneficiados que aprendían anatomía.

En los últimos siglos el avance en el estudio anatómico ha ido relacionado al avance de otras disciplinas científicas y tecnológicas que han ayudado a ello.
% http://www.bl.uk/learning/artimages/bodies/vesalius/renaissance.html
% http://www.sciencemuseum.org.uk/broughttolife/themes/understandingthebody/dead.aspx
% http://en.wikipedia.org/wiki/Anatomy\_Act\_1832
% http://en.wikipedia.org/wiki/Cadaver
% http://en.wikipedia.org/wiki/History\_of\_anatomy
% http://en.wikipedia.org/wiki/Hippocratic\_Corpus
% http://en.wikipedia.org/wiki/Galen