\section{Introducción} % Para los títulos
El arte podría definirse como una expresión de la creatividad humana que produce obras apreciadas por su belleza o emotividad y que es innata en el ser humano \footnote{Definición según el Diccionario Oxford: www.oxforddictionaries.com/definition/english/art}. Existe desde el origen del ser humano y  % (http://www.oxforddictionaries.com/definition/english/art)
a lo largo de los siglos ha ido evolucionando hasta el arte contemporáneo. Junto con las obras artísticas también se han ido transformando las representaciones anatómicas de obras pictóricas y escultóricas.

La anatomía de superficie es la ciencia que estudia las características anatómicas que pueden ser estudiadas mediante la vista y la palpación \footnote{Definición según Wikiradiography: www.wikiradiography.com/page/Surface+Anatomy}% (http://www.wikiradiography.com/page/Surface+Anatomy)
, es decir, la superficie del cuerpo. En el caso de la anatomía humana estudia desde las proporciones del cuerpo hasta puntos de referencia visibles en el exterior que están relacionados con órganos y partes internas, tanto en pose estática como en movimiento. % (http://www.princeton.edu/~achaney/tmve/wiki100k/docs/Superficial\_anatomy.html)
% %http://www.theodora.com/anatomy/surface\_anatomy\_index.html

%Como la materia principal de mi trabajo es la antomía de superficie relataré lo más relevante acerca de cómo ha evolucionado el conocimiento de esta materia hasta llegar al conocimiento actual.
\subsection{Historia de la anatomía}
A lo largo de la historia ha habido diferentes impedimentos que han provocado que hasta épocas muy recientes la anatomía del cuerpo humano fuese una gran desconocida, incluso entre aquellos que fundamentalmente trabajaban vinculados al cuerpo humano por distintas razones, como pueden ser médicos o artistas.

Desde el antiguo Egipto se ha estudiado la anatomía del ser humano, aunque no estaba a la altura del desarrollo médico que poseían, posiblemente por el gran respeto que profesaban a los cadáveres. Conocían algunas estructuras como el corazón, el cual consideraban el órgano central y lugar del pensamiento y los sentimientos, el hígado, el bazo, los riñones e incluso la existencia de vasos sanguíneos que transportaban distintos fluidos.

Los griegos fueron los grandes anatómicos de la antigüedad, de hecho, se recogen bastantes textos médicos en el Corpus Hipocraticum \footnote{Colección de trabajos médicos elaborados en la antigua Grecia. A pesar de su nombre no está comprobado que Hipócrates sea el autor de todos estos escritos, de hecho se desconoce el autor de la mayoría de ellos.}. En estos se describen la función de algunos órganos, así como del sistema músculo-esquelético y también la diferencia entre arterias y venas. Fueron los primeros en formar escuelas de anatomía e incluso los primeros en diseccionar cuerpos humanos \footnote{Ptolomeo I fue el primero que aprueba la utilización de cuerpos humanos para su disección y estudio anatómico. Durante su reinado y el de su dinastía fue posible para algunos anatomistas la realización de disecciones. El más importante es Herófilo considerado como "Padre de la anatomía".}, lo que permitió un conocimiento más avanzado del cuerpo humano. Sin embargo, esto no era lo habitual y normalmente utilizaban sus conocimientos de las disecciones animales para aplicarlos a la anatomía humana.

Galeno siguió con los estudios anatómicos mediante la disección de animales aportando además otros conocimientos que recibía gracias a su condición de médico entre los gladiadores donde podía vislumbrar cualquier tipo de herida. Sus estudios se encuentran recogidos en sus numerosos tratados. \footnote{Galeno con más de 500 tratados fue el autor de la antigüedad que más escritos elaboró.} En otras culturas como la india y la china la anatomía tampoco estaba más desarrollada que la occidental, que se basó durante varios siglos en los tratados galénicos.

El conocimiento anatómico del ser humano apenas avanzó en occidente durante la Edad Media, debido a la creencia cristiana de que el alma de un cuerpo diseccionado no podría ir al ``Cielo". Por ello las disecciones fueron un tema tabú durante esta época, y desapareció el interés por el cuerpo concentrándose éste en el alma. La cultura islámica poseía un conocimiento médico muy desarrollado para la época, sin embargo las disecciones de cadáveres humanos estaban prohibidas, por lo que su conocimiento de anatomía se basaba en las disecciones de animales y no era muy superior al de la Europa cristiana.

%Posteriormente, ya en la Edad moderna, se conceden permisos para diseccionar cadáveres de criminales como una parte más de la pena por su delito.

No es hasta Leonardo Da Vinci que se empezaron a hacer verdaderos progresos en la anatomía humana. Sus conocimientos se debieron a disecciones que llevó a cabo en cuerpos humanos y están plasmadas en una gran variedad de dibujos \footnote{El Hombre de Vitruvio es la más conocida de todas las ilustraciones elaboradas por Da Vinci (ver anexo \autoref{app:vitruvio}).}. Aunque el verdadero cambio de la anatomía galénica a una doctrina más moderna se dio en el siglo XVI gracias a Andrés Vesalio. Éste es el autor de "De Humani Corporis Fabrica", donde se recogen diversos dibujos de anatomía humana (ver anexo \autoref{app:vesalio}). Vesalio, junto con otros personajes influyentes de la época y posteriores como Miguel de Servet o William Harvey, provocó una auténtica revolución en la anatomía tradicional.

%Además, a partir de este momento los estudiantes de anatomía no tenían que saber latín para poder estudiar y lo podían hacer mediante las diversas representaciones que se estaban llevando a cabo.

Las disecciones empiezaron a ser más comunes, y se observaron varios problemas.

El primer problema se debía a la temperatura del ambiente. En aquella época no se conocía la manera de conservar un cuerpo. Es por esto que la disecciones se podían realizar exclusivamente en los meses de temperatura más baja, cuando el frío mantienía el cadáver por más tiempo. Aún así había que realizar la disección casi inmediatamente después de la muerte del individuo, pues los cuerpos no tardaban mucho en descomponerse.

El segundo problema provenía del aumento de las disecciones. Debido a los pocos cuerpos que permitían diseccionar y al aumento de estudiantes de anatomía se empezaron a robar tumbas para conseguir cadáveres, algunos incluso llegaron más lejos matando a personas para vender sus cuerpos \footnote{Son conocidos los asesinatos de West Port o de Burke y Hare que fueron llevados a cabo durante 1828 en Edimburgo. William Burke y William Hare asesinaron a dieciséis personas y vendieron sus cuerpos al Doctor Robert Knox, quien los utilizaba en sus clases de anatomía. Cuando se descubrieron los crímenes Hare testificó contra Burke quien fue condenado a muerte y después diseccionado.}.

Para evitar este último problema se elaboró la ley de 1832 que regulaba las donaciones de cuerpos a la ciencia y la obtención de licencias para obtener el permiso que permitía diseccionar cadáveres.

%Durante los siglos XVII y XVIII el campo de la anatomía se desarrolla tremendamente. Al principio se realizaban disecciones en plazas donde cualquiera podía atender a la explicación de un maestro, después, no obstante las disecciones pasan a realizarse en aulas siendo muchos menos los beneficiados que aprendían anatomía.

En los siglos XIX y XX el conocimiento anatómico adquirió su completa madurez gracias al avance de otras disciplinas científicas y tecnológicas que ayudaron a ello.
% http://www.bl.uk/learning/artimages/bodies/vesalius/renaissance.html
% http://www.sciencemuseum.org.uk/broughttolife/themes/understandingthebody/dead.aspx
% http://en.wikipedia.org/wiki/Anatomy\_Act\_1832
% http://en.wikipedia.org/wiki/Cadaver
% http://en.wikipedia.org/wiki/History\_of\_anatomy
% http://en.wikipedia.org/wiki/Hippocratic\_Corpus
% http://en.wikipedia.org/wiki/Galen
% Historia de la anatomía. Dr José Alfredo Sillau Gilone (en descargas 3 partes: libro de historia de la anatomía, historia anatomía e historia de la anatomía.)