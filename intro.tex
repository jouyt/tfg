\section{Introducción} % Para los títulos
El arte podría definirse como una expresión de la creatividad humana que produce obras apreciadas por su belleza o emotividad y que es innata en el ser humano. Existe desde el origen del ser humano y  % (http://www.oxforddictionaries.com/definition/english/art)
a lo largo de los siglos ha ido evolucionando hasta el arte contemporáneo. Junto con las obras artísticas también se han ido transformando las representaciones anatómicas de obras pictóricas y escultóricas.

La anatomía de superficie es la ciencia que estudia las características anatómicas que pueden ser estudiadas mediante la vista y la palpación% (http://www.wikiradiography.com/page/Surface+Anatomy)
, es decir, la superficie del cuerpo. En el caso de la anatomía humana estudia desde las proporciones del cuerpo hasta puntos de referencia visibles en el exterior que están relacionados con órganos y partes internas, tanto en pose estática como en movimiento. % (http://www.princeton.edu/~achaney/tmve/wiki100k/docs/Superficial\_anatomy.html)

En este trabajo con anatomía de superficie me refiero a las estructuras corporales que se pueden identificar visiblemente, obviando aquellas estructuras que se pueden apreciar mediante la palpación, puesto que el trabajo tratará de la comparación de varias obras pictóricas. Me centraré exclusivamente en cuatro obras pictóricas: El Cristo crucificado de Velázquez, el Cristo crucificado con dos donantes de el Greco, el Cristo amarillo de Gauguin y el Cristo de San Juan de la Cruz de Dalí.

Primero comenzaré explicando las proporciones del ser humano y sus cambios en el arte durante distintos períodos de la historia, ya que las proporciones del cuerpo humano también son importantes en cuanto a anatomía de superficie se refiere.

Analizaré cada obra pictórica individualmente, centrándome en el contexto histórico y en la anatomía superficial que podemos discernir en cada obra. Después desarrollaré una conclusión acerca se las similitudes y las diferencias que estas obras tienen entre sí.

Además, fuera del trabajo se incluirán varios anexos:
%http://www.theodora.com/anatomy/surface\_anatomy\_index.html