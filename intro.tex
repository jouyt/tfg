\section{Introducción} % Para los títulos
El arte podría definirse como una expresión de la creatividad humana que produce obras apreciadas por su belleza o emotividad y que es innata en el ser humano \footnote{Definición según el Diccionario Oxford: www.oxforddictionaries.com/definition/english/art}. Existe desde el origen del ser humano y  % (http://www.oxforddictionaries.com/definition/english/art)
a lo largo de los siglos ha ido evolucionando hasta el arte contemporáneo. Junto con las obras artísticas también se han ido transformando las representaciones anatómicas en las obras pictóricas y escultóricas.

La anatomía de superficie es la ciencia que estudia las características anatómicas que pueden ser estudiadas mediante la vista y la palpación \footnote{Definición según Wikiradiography: www.wikiradiography.com/page/Surface+Anatomy}% (http://www.wikiradiography.com/page/Surface+Anatomy)
, es decir, la superficie del cuerpo. En el caso de la anatomía humana estudia desde las proporciones del cuerpo hasta puntos de referencia visibles en el exterior que están relacionados con órganos y partes internas, tanto en pose estática como en movimiento. % (http://www.princeton.edu/~achaney/tmve/wiki100k/docs/Superficial\_anatomy.html)
% %http://www.theodora.com/anatomy/surface\_anatomy\_index.html

%Como la materia principal de mi trabajo es la antomía de superficie relataré lo más relevante acerca de cómo ha evolucionado el conocimiento de esta materia hasta llegar al conocimiento actual.
\subsection{Historia de la anatomía}
A lo largo de la historia ha habido diferentes impedimentos que han provocado que hasta épocas muy recientes la anatomía del cuerpo humano fuese una gran desconocida, incluso entre aquellos que fundamentalmente trabajaban vinculados al cuerpo humano por distintas razones, como pueden ser médicos o artistas.

Desde el antiguo Egipto se ha estudiado la anatomía del ser humano, aunque no estaba a la altura del desarrollo médico que poseían, posiblemente por el gran respeto que profesaban a los cadáveres. Conocían algunas estructuras como el corazón, el cual consideraban el órgano central y lugar del pensamiento y los sentimientos, el hígado, el bazo, los riñones e incluso la existencia de vasos sanguíneos que transportaban distintos fluidos.

Los griegos fueron los grandes anatómicos de la antigüedad, de hecho, se recogen bastantes textos médicos en el Corpus Hipocraticum \footnote{Colección de trabajos médicos elaborados en la antigua Grecia. A pesar de su nombre no está comprobado que Hipócrates sea el autor de todos estos escritos, de hecho se desconoce el autor de la mayoría de ellos.}. En estos se describen la función de algunos órganos, así como del sistema músculo-esquelético y también la diferencia entre arterias y venas. Fueron los primeros en formar escuelas de anatomía e incluso los primeros en diseccionar cuerpos humanos \footnote{Ptolomeo I fue el primero que aprueba la utilización de cuerpos humanos para su disección y estudio anatómico. Durante su reinado y el de su dinastía fue posible para algunos anatomistas la realización de disecciones. El más importante es Herófilo considerado como "Padre de la anatomía".}, lo que permitió un conocimiento más avanzado del cuerpo humano. Sin embargo, esto no era lo habitual y normalmente utilizaban sus conocimientos de las disecciones de animales para aplicarlos a la anatomía humana.

Galeno siguió con los estudios anatómicos mediante la disección de animales aportando además otros conocimientos que recibía gracias a su condición de médico entre los gladiadores donde podía vislumbrar cualquier tipo de herida. Sus estudios se encuentran recogidos en sus numerosos tratados. \footnote{Galeno con más de 500 tratados fue el autor de la antigüedad que más escritos elaboró.}

En otras culturas como la india y la china la anatomía tampoco estaba más desarrollada que la occidental, que se basó durante varios siglos en los tratados galénicos.

El conocimiento anatómico del ser humano apenas avanzó en occidente durante la Edad Media, debido a la creencia cristiana de que el alma de un cuerpo diseccionado no podría ir al ``Cielo". Por ello las disecciones fueron un tema tabú durante esta época, y desapareció el interés por el cuerpo concentrándose éste en el alma. La cultura islámica poseía un conocimiento médico muy desarrollado para la época, sin embargo las disecciones de cadáveres humanos estaban prohibidas, por lo que su conocimiento de anatomía se basaba en las disecciones de animales y no era muy superior al de la Europa cristiana.

%Posteriormente, ya en la Edad moderna, se conceden permisos para diseccionar cadáveres de criminales como una parte más de la pena por su delito.

No es hasta Leonardo Da Vinci que se empezaron a hacer verdaderos progresos en la anatomía humana. Sus conocimientos se debieron a disecciones que llevó a cabo en cuerpos humanos y están plasmadas en una gran variedad de dibujos \footnote{El Hombre de Vitruvio es la más conocida de todas las ilustraciones elaboradas por Da Vinci (ver anexo \autoref{app:vitruvio}).}. Aunque el verdadero cambio de la anatomía galénica a una doctrina más moderna se dio en el siglo XVI gracias a Andrés Vesalio. Éste es el autor de "De Humani Corporis Fabrica", donde se recogen diversos dibujos de anatomía humana (ver anexo \autoref{app:vesalio}). Vesalio, junto con otros personajes influyentes de la época y posteriores como Miguel de Servet o William Harvey, provocó una auténtica revolución en la anatomía tradicional.

%Además, a partir de este momento los estudiantes de anatomía no tenían que saber latín para poder estudiar y lo podían hacer mediante las diversas representaciones que se estaban llevando a cabo.

Las disecciones empezaron a ser más comunes, y se observaron varios problemas.

El primer problema se debía a la temperatura del ambiente. En aquella época no se conocía la manera de conservar un cuerpo. Es por esto que la disecciones se podían realizar exclusivamente en los meses de temperatura más baja, cuando el frío mantenía el cadáver por más tiempo. Aún así había que realizar la disección casi inmediatamente después de la muerte del individuo, pues los cuerpos no tardaban mucho en descomponerse.

El segundo problema provenía del aumento de las disecciones. Debido a los pocos cuerpos que  se permitían diseccionar y al aumento de estudiantes de anatomía se empezaron a profanar tumbas para conseguir cadáveres, algunos incluso llegaron más lejos matando a personas para vender sus cuerpos \footnote{Son conocidos los asesinatos de West Port o de Burke y Hare que fueron llevados a cabo durante 1828 en Edimburgo. William Burke y William Hare asesinaron a dieciséis personas y vendieron sus cuerpos al Doctor Robert Knox, quien los utilizaba en sus clases de anatomía. Cuando se descubrieron los crímenes Hare testificó contra Burke quien fue condenado a muerte y después diseccionado.}.

Para evitar este último problema se elaboró la ley de anatomía de 1832 (Reino Unido) que regulaba las donaciones de cuerpos a la ciencia y la obtención de licencias para obtener el permiso que permitía diseccionar cadáveres.

%Durante los siglos XVII y XVIII el campo de la anatomía se desarrolla tremendamente. Al principio se realizaban disecciones en plazas donde cualquiera podía atender a la explicación de un maestro, después, no obstante las disecciones pasan a realizarse en aulas siendo muchos menos los beneficiados que aprendían anatomía.

En los siglos XIX y XX el conocimiento anatómico adquirió su completa madurez gracias al avance de otras disciplinas científicas y tecnológicas que ayudaron a ello.
% http://www.bl.uk/learning/artimages/bodies/vesalius/renaissance.html
% http://www.sciencemuseum.org.uk/broughttolife/themes/understandingthebody/dead.aspx
% http://en.wikipedia.org/wiki/Anatomy\_Act\_1832
% http://en.wikipedia.org/wiki/Cadaver
% http://en.wikipedia.org/wiki/History\_of\_anatomy
% http://en.wikipedia.org/wiki/Hippocratic\_Corpus
% http://en.wikipedia.org/wiki/Galen
% Historia de la anatomía. Dr José Alfredo Sillau Gilone (en descargas 3 partes: libro de historia de la anatomía, historia anatomía e historia de la anatomía.)


\subsection{Muerte de Cristo}
La muerte de Cristo es una tema muy recurrido en la obra pictórica. En los siglos inmediatamente posteriores a la crucifixión no representaban tal cual al Cristo en la cruz, sino que lo hacían mediante otros símbolos y más tarde exclusivamente la cruz. Fue, sobre todo, a partir de la Edad Media (siglo V) cuando empezó a ser frecuente la imagen del Cristo crucificado.

Por otra parte, la iconografía cristiana varía de forma y estilo según las percepciones de la época artística en la que se desarrolla, en la que predomina un estilo determinado, pero en todas ellas ha tenido gran relevancia. De un joven imberbe se pasa a un hombre con barba. Ambas representaciones confluyen, pero artísticamente al final persiste la segunda. También se pasa de una visión de un Cristo triunfal a una visión más humana y menos divina de Cristo en el suplicio de la crucifixión.

Puesto que fue el emperador Constantino quien abolió la crucifixión en el siglo IV d.C, y no hay documentación ni representaciones acerca de ello hasta varios siglos después, es difícil saber cómo era, en la época romana, esta pena capital, que se utilizaba únicamente para los peores delincuentes.

Se sabe que aunque ha habido distintas formas de crucifixión, los romanos del periodo comprendido alrededor del nacimiento y la muerte de Cristo utilizaban la cruz \textit{commisa}, vulgarmente denominada cruz latina que es la más frecuentemente representada, o la cruz \textit{immissa}, en forma de T (ver anexo \autoref{app:crosses}). Éstas estaban formadas por dos tablones de madera: un poste vertical que se insertaba en el suelo, que recibe el nombre de \textit{stipes} y un travesaño horizontal denominado \textit{patibulum}. A veces en la mitad del poste vertical se insertaba un bloque de madera que servía de asiento. También se utilizaba, aunque probablemente posteriormente al tiempo en el que vivió Cristo, un tablón que colocado a la altura de los pies ejercía de apoyo para éstos.

Dependiendo de la altura de las cruces éstas se dividían en la \textit{cruz humilde}, la más habitualmente usada, cuya altura era de unos dos metros y la \textit{cruz sublime}, tan alta que los pies del condenado se encontraban aproximadamente a un metro del suelo. Se cree que Cristo pudo haber sido crucificado en una de estas últimas ya que necesitaron una caña para poder acercarle la esponja con vinagre cuando estaba al borde de la muerte (Mt 27, 48; Mc 15, 36; Jn 19, 29). En este aspecto, muchas de las representaciones que podemos observar de crucifixiones de Cristo podrían ser correctas.

Diversas fuentes dejan claro que los romanos utilizaban clavos para sujetar a los individuos crucificados a la cruz \footnote{Así como los evangelios atestiguan que Jesús fue clavado en la cruz, otros autores coetáneos se refieren también al uso de los clavos en la crucifixión. Ejemplos de ello son Séneca y Tranión. Además, las únicas reglas jurídicas que se conocen desde 1967 referentes a la crucifixión dan a entender que a los crucificados se les clavaba en la cruz.}. Además, la crucifixión variaba según la región y la inventiva del verdugo.

La creencia más extendida actualmente es la de que los clavos eran introducidos a la altura de la línea de flexión  de la muñeca, entre el cúbito y el radio y justo entre estos y los carpos o entre las dos filas de huesos metacarpianos. Se ha llegado a esta conclusión tras haber encontrado hallazgos arqueológicos de esa época que concuerdan con esta teoría y tras la realización de experimentos que apuntan a esta misma conclusión\footnote{El doctor Pierre Barbet realizó una serie de experimentos con cadáveres en los que se demostraba esta teoría.}. Anteriormente se creía que los clavaban en las manos porque así lo establece Juan en su Evangelio (Jn 20, 20-29), sin embargo ha de tenerse en cuenta que, antiguamente, la muñeca se encontraba en lo que se denominaba ``mano", que abarcaba hasta el brazo, lo que ha podido inducir a error durante siglos.

En cuanto a la técnica de clavado de los pies, no está claro si realizaban la inserción de los clavos juntando ambos pies y utilizado para ello un solo clavo, o la fijación de éstos se llevaba a cabo por separado mediante dos clavos. Probablemente los clavos eran insertados entre el segundo y tercer metatarsiano. En las representaciones pictóricas se emplea tanto la imagen de Cristo crucificado mediante tres clavos como mediante cuatro.

Teniendo en cuenta las dificultades en el conocimiento de la crucifixión de Cristo es lógico pensar que en la mayoría de la iconografía referente al tema Cristo es representado según el gusto y estilo de cada autor claramente ubicado en su época.

% Enciclopedia moderna, 11: diccionario universal de literatura, ciencias, artes, agricultura, industria y comercio. Francisco de Paula Mellado (Pág:805-811)
% http://www.frugalsites.net/jesus/crucifixion.htm
% http://mb-soft.com/believe/text/crucifix.htm
% LA CRUCIFIXIÓN Laura RODRÍGUEZ PEINADO (en documentos del TFG)
% http://www.shroud.com/bucklin2.htm

Si existe debate acerca de si Jesús murió verdaderamente en la cruz \footnote{Algunos autores como Margaret y Trevor Lloyd Davies, los mayores defensores, consideran la posibilidad de que Jesucristo no muriese verdaderamente en la cruz. Otro ejemplo de esta hipótesis se encuentra reflejada en el libro \textit{42 días, Análisis forense de la crucifixión y la resurrección de Jesucristo}, de Miguel Lorente.}, aún es mayor el debate existente sobre la presunta causa de su muerte.

Existen varias teorías al respecto:

1) Embolia Pulmonar

2) Rotura cardíaca

3) Trauma de suspensión

4) Asfixia

5) Herida perforante fatal

6) Shock

7) Síncope fatal

8) Arritmia cardíaca

9) Coagulopatía inducida por traumatismos.

Todas estas hipótesis han sido analizadas en profundidad, no llegando, sin embargo, a la conclusión acerca de cuál es la real. Lo que está claro, sin embargo, es que Cristo padeció terribles sufrimientos, los cuales en conjunto pudieron conducirle a su muerte, que se produjo hacia las tres de la tarde tras varias horas en la cruz.

% The Search for the Physical Cause of Jesus Christ's Death Author: W. Reid Litchfield Categories: God and Jesus Christ, Science Journal: 37:4 (en documentos de TFG)
% The crucifixion of Jesus: Review of hypothesized mechanisms of death and implications of shock and trauma-induced coagulopathy (en refworks)