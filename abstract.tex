\section{Abstract}
La muerte de Cristo, siendo una gran desconocida excepto por las distintas narraciones de los evangelios, es un tema muy recurrido en el arte. Analizando varias obras se aprecia un recorrido histórico de la representación anatómica y de la muerte de Cristo, siendo éstas variables según las las distintas creencias, percepciones sociales y religiosas e ideales de belleza de cada momento histórico.

Mediante el estudio de cuatro obras pictóricas se aprecian las variaciones en la anatomía de superficie de las figuras representadas, según el conocimiento anatómico de la época que actúa como principal factor influyente. Pudiendo examinar desde las proporciones seguidas por el artista, hasta las características menos humanas y más divinas con las que algunos autores nos fascinan.

Además, se pueden distinguir los diferentes percepciones de Cristo en varias épocas en cuanto a la representación de su gloria o su muerte, más humana, respectivamente.

\vspace{15pt}

\textbf{Palabras Clave:}
Anatomía de superficie, Cristo, crucifixión, obra pictórica, muerte.

