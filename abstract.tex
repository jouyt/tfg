\textbf{Resumen}

La muerte a la largo de la historia ha sido temida, respetada, alabada o evitada en la medida de lo posible. Incluso en la actual sociedad, %a pesar de ser conscientes de que a todos nos llegará la hora de morir, 
sigue siendo un tema tabú.

Aún así, la muerte de Cristo en la cruz, siendo una gran desconocida, excepto por las distintas narraciones de los evangelios, es un tema muy recurrido en el arte, siendo ampliamente representada desde la antigüedad.

Analizando tres obras pictóricas y una escultórica se aprecia un recorrido histórico de la representación anatómica y de la muerte de Cristo, siendo éstas variables según las distintas creencias, percepciones sociales y religiosas e ideales de belleza de cada momento histórico. En estas obras se observan las variaciones en la anatomía de superficie de las figuras representadas, según el conocimiento anatómico de la época en la que se realiza cada una. De aquí la importancia del análisis de la historia de la anatomía, sin la cual grandes obras no podrían haberse realizado.

%Mediante el estudio de tres obras pictóricas y una escultórica se observan las variaciones en la anatomía %superficial de las figuras representadas, según el conocimiento anatómico de la época en la que se realiza cada una. %está realizada cada una de ellas. %que actúa como principal factor influyente. 
%De aquí la importancia del análisis de la historia de la anatomía, sin la cual grandes obras %pictóricas y escultóricas 
%no podrían haberse realizado.

Además, cada autor refleja de distinta forma la muerte de Cristo, obedeciendo a su propio conocimiento y a los principios de la época, desde un Cristo que no posee un rasgo de muerte en su apariencia, hasta otro que, en la búsqueda de un total realismo, refleja %lo más científicamente posible todas y cada una de las lesiones de este, además de 
todas las características presentes en la muerte. %de cualquier persona.

%Se pueden examinar desde las proporciones seguidas por el artista, hasta las características menos humanas y más divinas con las que algunos autores nos fascinan.

Asimismo, se pueden distinguir, en varias épocas, las diferentes percepciones de Cristo en cuanto a la representación de su gloria, como hijo de Dios y Dios, o su muerte, más humana, respectivamente.

%\vspace{10pt}

\textbf{Palabras Clave:}
Anatomía superficial, Cristo, muerte, crucifixión, obra de arte.

\vspace{10pt}

\textbf{Abstract}

Throughout history death has been feared, respected, or avoided if possible. Even in today's society it is still a taboo subject.

Even so, the death of Christ, being a great unknown except for all the accounts in the Gospels, is one of the more recurrent topics in art, being widely represented since antiquity.

Analyzing three paintings and one sculpture, it is noticeable a historical path of the anatomical representation and of the death of Christ, those of whom are changeable according to the different beliefs, social and religious perceptions and beauty ideals from each period of time.
%By the study of three paintings and one sculpture
In these artworks, it is observable the variations on the surface anatomy of the represented figures, according to the anatomical knowledge in the epoch in which each is performed, %which acts as the main influential factor. 
Hence the importance of analyzing 
the history of anatomy, without which large works of art%paintings and sculptures 
could not have been done.

In addition, each author reflects differently the death of Christ, in accordance with their own knowledge and the principles of the time, from a Christ who does not have any feature of death to another who, searching for total realism, reflects %as scientifically as possible %each and everyone of Christ injuries as well as 
all the features present in %anyone's 
death.

%It can be exhaminated from the proportions given by the artist until the less human features and the more divine ones some authors fascinate us with.

Furthermore, the different perceptions of Christ can be distinguished along the time, referring to his glory representation, as son of God and God, or even his death, more human, respectively.

%\vspace{15pt}

\textbf{Key words:}
Surface anatomy, Christ, death, crucifixion, artwork.