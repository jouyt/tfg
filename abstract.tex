\textbf{Resumen}

La muerte a la largo de la historia ha sido temida, respetada, alabada o evitada en la medida de lo posible e incluso sigue siendo un tema tabú en la sociedad actual.

La muerte de Cristo en la cruz aún siendo una gran desconocida, excepto por las distintas narraciones de los evangelios, es un tema muy recurrido en el arte y ampliamente representado desde la antigüedad.

Analizando tres obras pictóricas y una escultórica se aprecia un recorrido histórico de la representación anatómica y de la muerte de Cristo, variando según las distintas creencias, percepciones sociales y religiosas e ideales de belleza de cada momento histórico. % En estas obras se observan las variaciones en la anatomía de superficie de las figuras representadas, según el conocimiento anatómico de la época en la que se realiza cada una. De aquí la importancia del análisis de la historia de la anatomía, sin la cual grandes obras no podrían haberse realizado.
Según el conocimiento anatómico de la época en la que se realiza cada una de las obras, se observan variaciones en la anatomía de superficie de las figuras representadas. Muchas de las grandes obras no podrían haberse realizado sin este conocimiento, justificando así el análisis de la historia de la anatomía.

Además, cada autor refleja de distinta forma la muerte de Cristo, obedeciendo a su propio conocimiento y a los principios de la época, desde un Cristo que no posee un rasgo de muerte en su apariencia, hasta otro que, en la búsqueda de un total realismo, refleja todas las características presentes en la muerte.

Asimismo, se pueden distinguir, en varias épocas, las diferentes percepciones de Cristo en cuanto a la representación de su gloria, como hijo de Dios y Dios, o su muerte, más humana, respectivamente.


\textbf{Palabras Clave:}

Anatomía superficial, Cristo, muerte, crucifixión, obra de arte.

\vspace{10pt}

\textbf{Abstract}

Throughout history death has been feared, respected, or avoided if possible and even in today's society it is still a taboo subject.

Although the death of Christ is a great unknown topic, except for all the accounts in the Gospels, it is also one of the more recurrent topics in art, being widely represented since antiquity.
%Even so, the death of Christ, being a great unknown except for all the accounts in the Gospels, is one of the more recurrent topics in art, being widely represented since antiquity.

A historical path of the anatomical representation of the death crist will be observed through the analisys of three painting and one sculpture. Each of them according to the different beliefs, social and religious perceptions and beauty ideals from each period of time.
%Through the analisys of three paintings and one sculpture, it is noticeable a historical path of the anatomical representation and of the death of Christ, those of whom are changeable according to the different beliefs, social and religious perceptions and beauty ideals from each period of time.
In these artworks, it is observable the variations on the surface anatomy of the represented figures, according to the anatomical knowledge in the epoch it belongs to. Many great artworks could not have been done without this knowledge, thereby justifying the analysis of the history of the anatomy. 
%Hence the importance of analyzing the history of anatomy, without which large works of art could not have been done.

Each author reflects differently the death of Christ, in accordance with their own knowledge and the principles of the time, from a Christ who does not have any feature of death to another who, searching for total realism, reflects all the features of death.

Furthermore, the different perceptions of Christ can be distinguished along the time, referring to his glory representation, as son of God and God, or even his death, more human, respectively.


\textbf{Key words:}
Surface anatomy, Christ, death, crucifixion, artwork.