\textbf{Resumen}

La muerte de Cristo, siendo una gran desconocida excepto por las distintas narraciones de los evangelios, es un tema muy recurrido en el arte. Analizando varias obras se aprecia un recorrido histórico de la representación anatómica y de la muerte de Cristo, siendo éstas variables según las las distintas creencias, percepciones sociales y religiosas e ideales de belleza de cada momento histórico.

Mediante el estudio de cuatro obras pictóricas se observan las variaciones en la anatomía de superficie de las figuras representadas, según el conocimiento anatómico de la época que actúa como principal factor influyente. Pudiendo examinar desde las proporciones seguidas por el artista, hasta las características menos humanas y más divinas con las que algunos autores nos fascinan.

Además, se pueden distinguir los diferentes percepciones de Cristo en varias épocas en cuanto a la representación de su gloria, como hijo de Dios y Dios, o su muerte, más humana, respectivamente.

\vspace{15pt}

\textbf{Palabras Clave:}
Anatomía de superficie, Cristo, crucifixión, obra pictórica, muerte.

\vspace{30pt}

\textbf{Abstract}

The death of Christ, being a great unknown except for all the accounts in the Gospels, is one of the more recurrent topics in art. Analizing some artworks it is noticeable a historical path of the anatomical representation and of th e death of Christ, those of whom are changeable according to th different beliefs, social and religious perceptions and beauty ideals from each period of time.

By the study of four works of art, it is observable the variations on the surface anatomy of the represented figures, according to the anatomical knowledgement in each epoch, which acts as the main influential factor. It can be exhaminated from the proportions given by the artist until the less human features and the more divine ones some authors fascinate us with.

Furthermore, the different perceptions of Christ can be distinguished  along the time, refering to his glory representation, as son of God and God, or even his death, more human, respectively.

\vspace{15pt}

\textbf{Key words:}
Surface anatomy, Christ, crucifixion, artwork, death.