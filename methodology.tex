\section{Metodología}
En este trabajo con anatomía de superficie me refiero a las estructuras corporales que se pueden identificar visiblemente, obviando aquellas estructuras que se pueden apreciar mediante la palpación, puesto que el trabajo tratará de la comparación de varias obras pictóricas. Me centraré exclusivamente en cuatro obras pictóricas: El Cristo crucificado de Velázquez, el Cristo crucificado de el Greco, la lamentación sobre Cristo muerto de Mantegna % el Cristo amarillo de Gauguin 
y el Cristo de San Juan de la Cruz de Dalí.

%Primero comenzaré explicando las proporciones del ser humano y sus cambios en el arte durante distintos períodos de la historia, ya que las proporciones del cuerpo humano también son importantes en cuanto a anatomía de superficie se refiere.

Analizaré cada obra pictórica individualmente, centrándome en el contexto histórico y en la anatomía superficial que podemos discernir en cada obra. También estudiaré las caracerísticas presentes en las obras en relación a la muerte y la agonía de Cristo. Después desarrollaré una conclusión acerca se las similitudes y las diferencias que estas obras muestran entre sí.

Además, fuera del trabajo se incluirán varios anexos: