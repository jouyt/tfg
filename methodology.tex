\section{Metodología}
%TODO Hacer que el tiempo de esta sección esté acorde al utilizado anteriormente
En este trabajo con anatomía de superficie me refiero a las estructuras corporales que se pueden identificar visiblemente, obviando aquellas estructuras que se pueden apreciar mediante la palpación, puesto que el trabajo trata de la comparación de varias obras pictóricas. Me centro exclusivamente en cuatro obras pictóricas: El Cristo crucificado de Velázquez, el Cristo crucificado de El Greco, la lamentación sobre Cristo muerto de Mantegna % el Cristo amarillo de Gauguin 
y el Cristo de San Juan de la Cruz de Dalí.

%Primero comenzaré explicando las proporciones del ser humano y sus cambios en el arte durante distintos períodos de la historia, ya que las proporciones del cuerpo humano también son importantes en cuanto a anatomía de superficie se refiere.

Antes de examinar las obras y puesto que es el tema principal de las obras, mencionaré algunos puntos básicos acerca de la crucifixión y muerte de Cristo.

A continuación, analizaré cada obra pictórica individualmente, centrándome en el contexto histórico y en la anatomía superficial que podemos discernir en cada obra. También estudiaré las características presentes en las obras en relación a la muerte y la agonía de Cristo.

Por último, desarrollaré una conclusión acerca se las similitudes y las diferencias que estas obras muestran entre sí, así como del grado de fidelidad en relación a la época representada en ellas.

Además, fuera del trabajo se incluirán varios anexos:

\vspace{12pt}

Para redactar el documento se empleará un editor de texto llamado TexStudio, preparado para utilizar la herramienta de procesamiento de texto llamada Latex. Como fuentes de información principal en la realización del trabajo se utilizará el Encore, accesible mediante la página web de la biblioteca de la UPV-EHU, el cual está adscrito a numerosas bases de datos. Además, varios buscadores serán de gran utilidad en el ámbito que estudiaré: Google Academic y Google Art.

Una vez conseguida la información que podría ayudarme en la realización de mi TFG, desecharé aquella información que por distintas razones diverge de lo que se busca y me centraré en aquellos documentos, artículos y libros que considere de interés para el trabajo. Tras extensas lecturas y mucha síntesis lograré cumplir lo propuesto en los objetivos del trabajo.

Los anexos contenidos fuera del trabajo y los pies de página reflejarán información complementaria útil en la compresión del trabajo, que aún sin ser imprescindible, es una información que considero de interés.