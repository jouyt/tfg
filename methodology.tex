\section{Metodología}
%TODO Hacer que el tiempo de esta sección esté acorde al utilizado anteriormente
En este trabajo con anatomía de superficie nos referiremos a las estructuras corporales que se pueden identificar visiblemente, obviando aquellas estructuras que se pueden apreciar mediante la palpación, puesto que el trabajo tratará de la comparación de varias obras pictóricas y escultóricas. Me centraré exclusivamente en cuatro obras pictóricas: Lamentación sobre Cristo muerto de Mantegna, Cristo crucificado de Velázquez,  % el Cristo amarillo de Gauguin 
Cristo de San Juan de la Cruz de Dalí y Cristo yacente de la Síndone de Miñarro.

%Primero comenzaré explicando las proporciones del ser humano y sus cambios en el arte durante distintos períodos de la historia, ya que las proporciones del cuerpo humano también son importantes en cuanto a anatomía de superficie se refiere.

Antes de examinar las obras, y puesto que es el tema principal de éstas, se mencionarán algunos puntos básicos acerca de la crucifixión y muerte de Cristo.

A continuación, se analizará cada obra pictórica individualmente, centrándonos en el contexto histórico y en la anatomía superficial que podemos discernir en cada obra. También se estudiarán las características presentes en las obras en relación a la agonía y la muerte de Cristo.

Por último, se desarrollará una conclusión acerca de las similitudes y las diferencias que estas obras muestran entre sí, así como del grado de fidelidad en relación a la época representada en ellas.

Además, como complemento al trabajo se incluirán varios anexos:

\vspace{12pt}

Para redactar el documento se empleará un editor de texto llamado TexStudio, preparado para utilizar la herramienta de procesamiento de texto llamada Latex. Como fuente de información principal en la realización del trabajo se utilizará el Encore, accesible mediante la página web de la biblioteca de la UPV-EHU, el cual está adscrito a numerosas bases de datos. Además, varios buscadores serán de gran utilidad en el ámbito a estudiar: Google Academic y Google Art.

Una vez conseguida la información que podría ayudarnos en la realización del TFG, se desechará aquella información que por distintas razones diverge de lo que se busca y nos centraremos en aquellos documentos, artículos y libros que consideramos de interés para el trabajo. Tras extensas lecturas y mucha síntesis se conseguirá cumplir lo propuesto en los objetivos del trabajo.

Los anexos y los pies de página reflejarán información complementaria útil en la compresión del trabajo que, aún sin ser imprescindible, es una información que es considerada de interés.