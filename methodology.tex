\section{Metodología}
En este trabajo con anatomía de superficie me refiero a las estructuras corporales que se pueden identificar visiblemente, obviando aquellas estructuras que se pueden apreciar mediante la palpación, puesto que el trabajo trata de la comparación de varias obras pictóricas. Me centro exclusivamente en cuatro obras pictóricas: El Cristo crucificado de Velázquez, el Cristo crucificado de el Greco, la lamentación sobre Cristo muerto de Mantegna % el Cristo amarillo de Gauguin 
y el Cristo de San Juan de la Cruz de Dalí.

%Primero comenzaré explicando las proporciones del ser humano y sus cambios en el arte durante distintos períodos de la historia, ya que las proporciones del cuerpo humano también son importantes en cuanto a anatomía de superficie se refiere.

Antes de examinar las obras, menciono algunos puntos básicos acerca de la crucifixión y muerte de Cristo, puesto que es el tema principal de las obras.

A continuación, analizaré cada obra pictórica individualmente, centrándome en el contexto histórico y en la anatomía superficial que podemos discernir en cada obra. También estudiaré las características presentes en las obras en relación a la muerte y la agonía de Cristo.

Por último, desarrollaré una conclusión acerca se las similitudes y las diferencias que estas obras muestran entre sí, así como del grado de fidelidad en relación a la época representada en ellas.

Además, fuera del trabajo se incluirán varios anexos:

\vspace{12pt}

Para redactar lo anteriormente descrito he empleado un editor de texto llamado Latex. Como fuentes de información en la realización del trabajo he utilizado el Encore, mediante la página web de la biblioteca de la UPV-EHU, el cual está adscrito a numerosas bases de datos. Además, varios buscadores han sido de gran utilidad: el google académico y el google art.

Una vez conseguida la información que podría ayudarme en la realización de mi TFG, deseché aquella información que por distintas razones divergía de lo que estaba buscando y me centré en aquellos documentos, artículos y libros que me interesaban. Tras extensas lecturas y muchas síntesis logré hacer aquello que me propuse en los objetivos del trabajo.

Los anexos contenidos fuera del trabajo y los pies de página reflejan información complementaria útil en la compresión del trabajo, por lo que aún sin ser imprescindible, es una información necesaria.