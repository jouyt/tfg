\section{Análisis de la obra pictórica: Cristo crucificado de Velázquez } 

Arte, estilo: Barroco

Cronología: 1632

Lugar: Museo del Prado, Madrid,España

Autor: Diego Velázquez

Título: Cristo crucificado

Función: Aunque se encontraba en el Convento de la encarnación de San Plácido, no está claro donde estuvo los primeros años tras su realización ni la función específica que desarrollaba. Posteriormente se tiene constancia de que estuvo en la sacristía de este convento hasta alrededor del 1804, año en el que Godoy lo compra al convento. Pasa después por distintas manos, llegando finalmente al Museo de Prado en 1829, donde permanece actualmente. % El Cristo crucificado de velazquez, trasfondo histórico religioso. % http://www.museodelprado.es/coleccion/galeria-on-line/galeria-on-line/obra/cristo-crucificado-1/

\textbf{Anatomía de superficie}
En el cuadro se puede observar la parte izquierda de la cara, la parte derecha está cubierta por el cabello que cae sobre ella. En cuanto a aquello que podemos ver, se trata de una cara inexpresiva con barba que cubre parte de las mejillas y el mentón. Bajo esta barba se puede apreciar la forma del hueso de la manbíbula (cuerpo, rama y ángulo). La mucosa de los labios es apenas visible entre el vello, aunque sí se intuyen incluso las comisura labial izquierda. La nariz se presenta en el centro de la cara de forma piramidal. La estructura de la nariz la proporcionan los huesos nasales que articulan entre sí, con el hueso frontal(por donde se une a la frente) y con el hueso maxilar (por el que se une  a las mejillas). La parte más prominente de la nariz es el cartílago nasal que separa las dos fosas nasales. Se aprecia el espacio del surco naso-labial, aunque éste no se ve directamente, porque al igual que el mentón, está cubierto de vello. Sin embargo, sí se distingue la aleta nasal izquierda y el septo nasal, a la vez que se puede intuir la narina izquierda de la nariz. Se puede ver la prominencia de la mejilla, formada por el hueso cigomático. Se puede también vislumbrar una pequeña arruga, la línea naso-labial. Entre la nariz y la frente se encuentra una depresión denominada nasión. La frente, formada por la convexidad lisa del hueso frontal, apenas es visible puesto que está tapada por la corona de espinas. Sus bordes inferiores (arcos supraciliares) dan lugar al borde superior de cada órbita. Además estos dos arcos se unen mediante un puente denomindo gabela. Superficialmente a estas elevaciones se encuentran las cejas. Medialmente el globo ocular está limitado por los huesos maxilar y frontal que articulan entre ellos, lateralmente está limitado por los huesos frontal y cigomático e inferiormente por el hueso maxilar. El globo ocular izquierdo no es visible, solo se adivina debajo del párpado superior que lo cubre, en el que además se puede observar una pequeña arruga. Sí se puede vislumbrar las uniones medial y lateral entre los párpados superior e inferior, que limitarían la hendidura palpebral en caso de estar abiertos. En sus bordes se encuentran las pestañas, manifestadas en la obra mediante una línea más oscura en el borde de los párpados. Ambas orejas están tapadas por el cabello, por lo que sus partes son visibles.
El resto de la cabeza está formada por la bóveda craneal, la cual se insinua debajo del cabello. La cara lateral de la bóveda la componen los huesos frontal, parietal, occipital y temporal.
El cuello no es visible, al estar la cabeza caída hacia abajo, ésta tapa las estructuras anatómicas del cuello.

En la imagen se puede observar el torso desnudo de Cristo crucificado, en el que se aprecian una serie de estructuras anatómicas. 

Al tratarse de un individuo delgado se puede intuir la caja torácico perfectamente. Vislumbramos, a duras penas, la clavícula izquierda que articula con el esternón (manubrio), la derecha, no es visible, puesto que está tapada por el cabello. El borde costal inferior formado por las seis últimas costillas es fácilmente visible, al igual que la depresión en la que se encuentra el cuerpo del esternón y el contorno de algunas costillas (probablemente la cuarta, quinta, sexta y séptima), sobre todo las del lado derecho, ya que la sombra del propio cuadro nos  impide ver bien las del izquierdo. La escotadura esternal, también se puede intuir sobre la cara superior del manubrio, sin embargo, hay que dejarse llevar un poco por la imaginación, ya que, de nuevo la sombra del cuadro nos impide su correcta visualización. No se observa la apófisis xifoides, que está cubierta, por el recto anterior del abdomen, lo que se considera normal.

En cuanto a los músculos podemos vislumbrar el pectoral mayor, que inserta en el borde interno de la clavícula, en los cartílagos de las cinco costillas superiores y en el húmero, formando así el pliegue anterior de la axila, muy claro en la imagen. El músculo deltoides se puede contemplar recubriendo la articulación del hombro. Éste inserta en clavícula y escápula.

\subsubsection{Pared torácica lateral}
Limitada inferiormente por el borde inferior de la caja costal y superiormente por la axila.
El músculo serrato anterior se inserta en el borde anterior interno de la escápula y en las ocho costillas superiores, formando ocho facículos que pueden verse en una persona delgada junto con sus interdigitaciones con el músculo oblicuo externo en las cuatro costillas medias.
La glándula mamaria femenina se extiende de la segunda costilla a la séptima y desde el borde externo del esternón hasta la pared axilar anterior. Descansa sobre los músculos  pectoral mayor, serrato anterior y oblicuo externo. El tamaño de la mama y la altura del pezón son variables en cada mujer. En el varón el pezón se encuentra en el cuarto espacio intercostal.

\subsubsection{Axila}
Es una cavidad rellena de grasa que separa el brazo del tórax. Su pared interna está formada por las seis costillas superiores y las paredes anterior(pectoral mayor) y posterior (músculo redondo mayor) convergen lateralmente sobre la porción superior de la diáfisis humeral.


Colores: primarios, secundarios, fríos, cálidos, paleta de colores.

Línea, dibujo

Luz

Perspectiva

Movimiento o quietud

Proporciones

Iconografía

Resolución del tema

Momento de la narración elegido
