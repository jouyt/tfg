\section{Análisis de la obra pictórica: Cristo crucificado de Velázquez } 

Arte, estilo: Barroco

Cronología: 1632

Lugar: Museo del Prado, Madrid,España

Autor: Diego Velázquez

Título: Cristo crucificado

Función: Aunque se encontraba en el Convento de la encarnación de San Plácido, no está claro donde estuvo los primeros años tras su realización ni la función específica que desarrollaba. Posteriormente se tiene constancia de que estuvo en la sacristía de este convento hasta alrededor del 1804, año en el que Godoy lo compra al convento. Pasa después por distintas manos, llegando finalmente al Museo de Prado en 1829, donde permanece actualmente. % El Cristo crucificado de velazquez, trasfondo histórico religioso. % http://www.museodelprado.es/coleccion/galeria-on-line/galeria-on-line/obra/cristo-crucificado-1/

\textbf{Anatomía de superficie}
En esta obra Velázquez nos cautiva con una figura con una postura serena y de proporciones y anatomía estudiadas. Responde a un modelo de siete cabezas y media. Su longitud con los brazos extendidos es igual a su altura y el ángulo que forman entre ellos es de 113º. La cadera esta ligeramente inclinada hacia el lado derecho, dando la sensación de apoyo del peso en el lado izquierdo, lo que se podría interpretar como un ligerísimo contrapposto clásico.
En el cuadro se puede observar la parte izquierda de la cara, la parte derecha está cubierta por el cabello que cae sobre ella. En cuanto a aquello que podemos ver, se trata de una cara inexpresiva con barba que cubre parte de las mejillas y el mentón. Bajo esta barba se puede apreciar la forma del hueso de la manbíbula (cuerpo, rama y ángulo). La mucosa de los labios es apenas visible entre el vello, aunque sí se intuyen incluso la comisura labial izquierda. La nariz se presenta en el centro de la cara de forma piramidal. La estructura de la nariz la proporcionan los huesos nasales que articulan entre sí, con el hueso frontal(por donde se une a la frente) y con el hueso maxilar (por el que se une  a las mejillas). La parte más prominente de la nariz es el cartílago nasal que separa las dos fosas nasales. Se aprecia el espacio del surco naso-labial, aunque éste no se ve directamente, porque al igual que el mentón, está cubierto de vello. Sin embargo, sí se distingue la aleta nasal izquierda y el septo nasal, a la vez que se puede intuir la narina izquierda de la nariz. Se puede también vislumbrar una pequeña arruga, la línea naso-labial. Se puede ver la prominencia de la mejilla, formada por el hueso cigomático. Entre la nariz y la frente se encuentra una depresión denominada nasión. La frente, formada por la convexidad lisa del hueso frontal, apenas es visible puesto que está tapada por la corona de espinas. Sus bordes inferiores (arcos supraciliares) dan lugar al borde superior de cada órbita. Además estos dos arcos se unen mediante un puente denomindo gabela. Superficialmente a estas elevaciones se encuentran las cejas. Medialmente el globo ocular está limitado por los huesos maxilar y frontal que articulan entre ellos, lateralmente está limitado por los huesos frontal y cigomático e inferiormente por el hueso maxilar. El globo ocular izquierdo no es visible, solo se adivina debajo del párpado superior que lo cubre, en el que además se puede observar una pequeña arruga. Sí se puede vislumbrar las uniones medial y lateral entre los párpados superior e inferior, que limitarían la hendidura palpebral en caso de estar abiertos. En sus bordes se encuentran las pestañas, manifestadas en la obra mediante una línea más oscura en el borde de los párpados. Ambas orejas están tapadas por el cabello, por lo que sus partes no son visibles.
El resto de la cabeza está formada por la bóveda craneal, la cual se insinua debajo del cabello. La cara lateral de la bóveda la componen los huesos frontal, parietal, occipital y temporal.
El cuello no es visible, al estar la cabeza caída hacia abajo, ésta tapa las estructuras anatómicas del cuello.

En la imagen se puede observar el torso desnudo de Cristo crucificado, en el que se aprecian una serie de estructuras anatómicas. 

Al tratarse de un individuo delgado se puede intuir la caja torácico perfectamente. Vislumbramos, a duras penas, la clavícula izquierda que articula con el esternón (manubrio) y la primera costilla, la derecha, no es visible, puesto que está tapada por el cabello. El borde costal inferior formado por las seis últimas costillas es fácilmente visible, al igual que la depresión en la que se encuentra el cuerpo del esternón y el contorno de algunas costillas (probablemente la cuarta, quinta, sexta y séptima), sobre todo las del lado derecho, ya que la sombra del propio cuadro nos  impide ver bien las del izquierdo. La escotadura esternal, también se puede intuir sobre la cara superior del manubrio, sin embargo, hay que dejarse llevar un poco por la imaginación, ya que, de nuevo la sombra del cuadro nos impide su correcta visualización. No se observa la apófisis xifoides, que está cubierta, por el recto anterior del abdomen, lo que se considera normal.

En cuanto a los músculos podemos vislumbrar el pectoral mayor, que inserta en el borde interno de la clavícula, en los cartílagos de las seis costillas superiores y en el húmero, formando así el pliegue anterior de la axila, muy claro en la imagen, junto con el músculo deltoides, que se puede contemplar recubriendo la articulación del hombro. Lateralmente no se observan claramente los músculos. El serrato anterior que inserta en el borde anterior interno de la escápula, y en las ocho costillas superiores, junto con sus interdigitaciones con el músculo oblicuo externo podrían ser visibles en un individuo delgado y musculoso como el Cristo que representa este cuadro, sin embargo no es fácil de reconocer. Sí se puede apreciar el músculo dorsal ancho, proveniente de la espalda e insertado en el ángulo inferior de la escápula. Su tendón estrecho envuelve al músculo redondo mayor que forma el pliegue posterior de la axila. Este va desde el borde lateral de la escápula hasta el surco bicipital del húmero. El borde lateral de la escápula puede percibirse a través de la masa del redondo mayor.
Los pezones se encuentran en el Cristo de la obra alrededor del cuarto espacio intercostal, lo que es anatómicamente correcto.

En el abdomen, que delimita superiormente con el reborde costal, que como ya he dicho es claramente visible en esta obra, se encuentran los músculos abdominales, que se pueden apreciar, aunque no muy marcados. Estos músculos forman una vaina fibrosa a cada lado de la línea media que se unen en esta, formando la línea alba. La figura no está suficientemente musculada para poder apreciar las tres inserciones tendinosas que se encontrarían encima del ombligo, a duras penas podríamos localizar una de ellas. Sí se divisa bastante bien la línea semilunar, que marca el borde lateral del músculo recto del abdomen. La cresta ilíaca si se aprecia, sobre todo en el lado derecho, que es el lado hacia el que está ligeramente inclinada la cadera.
El músculo mas superficial de los músculos laminares del abdomen es el oblicuo externo , pero como ya he comentado anteriormente no se ve claramente, y aunque sepamos que tiene que estar ahí, sus interdigitaciones con el serrato anerior no son visibles en el cuadro. Este músculo se inserta superiormente en las ocho costillas inferiores e inferiormente en la sínfisis de púbis, tuberosidad púbica, espina ilíaca anterosuperior y cresta ilíaca. Entre la espina y la tuberosidad púbica gira sobre sí mismo y forma el ligamento inguinal. La depresión que suele haber a la altura de ese ligamento, y que suele ser más visible en hombre, se puede observar perfectamente en la obra pictórica.
Las estructuras pélvicas no son visibles, debido a que el autor coloca encima un paño blanco con el que las tapa.

Las extemidades superiores se encuentran extendidas cada una hacia un costado de la cruz: Los músculos son visibles claramente. Antes de empezar con los músculos del brazo propiamente dichos, remarcaré algúno que se encuentra sustentando la articulación del hombro. El múculo trapecio sí se observa, este lateralmente se inserta en el tercio externo de la clavícula, el acromion y la espina de la escápula. Estas estructuras no son visibles anteriormente en la posición en la que se encuentra la figura analizada (brazos extendidos hacia los lados), sin embardo podemos apreciar un hueco en la unión de este músculo a las estructuras previamente citadas. El músculo que no nos deja ver esas estructuras es el deltoides, que también se ve fácilmente redondeando la articulación del hombro por su parte superior, cubriendo la epífisis proximal del húmero. Se inserta medialmente, al igual que el trapecio, en la clavícula, el acromion y la espina de la escápula y lateralmente en el tubérculo deltoideo de la cara lateral del húmero. Es el principal abductor del brazo por lo que tiene gran relevancia en esta posición.
En el brazo se observan tanto el bíceps como el tríceps. El bíceps forma una prominencia en la parte anterior del brazo. Tiene dos cabezas, de las cuales la corta inserta en la apófisis coracoides y la larga en la cara superior de la fosa glenoidea de la escápula. Distalmente se inserta en la tuberosidad bicipital del radio, pasando por el medio de la fosa cubital.
En ambos brazos podemos distinguir la fosa cubital. Esta está limitada superiormete por la línea que va desde la epitróclea hasta el epicóndilo del húmero, lateralmente por el músculo braquiorradial y medialmente por el pronador redondo.
Los huesos de la articulación del codo suelen ser evidentes. En esta obra no se pueden distinguir claramente las distintas estructuras debido a la posición de los brazos (se verían mejor desde la cara posterior). Lo único que se aprecia de forma clara es la epitróclea del húmero.
Los brazos se encuentran en total supinación y extensión. La extensión la realiza el tríceps junto con la contribución de los músculos del origen del extensor común. Y de la supinación se encargan el bíceps y el supinador.
\subsubsection{Visión anterior del antebrazo}
El músculo pronador se inserta sobre la cara anterior de la epitróclea y en la región central del radio. Es un pronador del antebrazo. Los tendones de los músculos flexor carpi radialis, palmar mayor, flexor superficial de los dedos y flexor carpi ulnaris pasan a la mano. Son también flexores débiles del codo.
El músculo braquiorradial se inserta proximalmente  en los dos tercios superiores de la cresta supracondílea del húmero y distalmete en la cara lateral de la epífisis distal del radio. Este músculo prona o supina el antebrazo.


\subsubsection{Movimientos de las articulaciones del codo y radio-cubital}
El codo es flexionado por los músculos bíceps y braquial, ayudados por el braquiorradial y los músculos del origen del flexor común (flexor carpi radialis, palmar mayor, flexor superficial de los dedos y flexor carpi ulnaris). La extensión la realiza el tríceps junto con la contribución de los músculos del origen del extensor común.
En posición anatómica la posición del antebrazo está en total supinación. La epífisis distal del radio puede hacer la rotación interna anterior al cúbito , a lo largo de 180 grados, dejando el dorso de la mano anterior y la palma posterior, esta es la posición de pronación completa. El bíceps y el supinador supinan y el pronador redondo y el pronador cuadrado pronan.
En la posición anatómica la apófisis estiloides del radio y del cúbito son palpables e incluso visibles. En completa pronación la apófisis estiloides radialsigue siendo palpable, pero ahora medialmente. La apófisis estiloides del cúbito sufre un desplazamiento lateral y edja de ser palpable. En esta posición el hueso más fácilmente palpable es la cabeza del cúbito.

\subsubsection{Visión anterior de la muñeca y de la mano}Cuando se flexiona la muñeca contra resistencia, tres tendones(flexor carpi radialis, palmar mayor y flexor carpi ulnaris) sobresalen. El flexor carpi raialis se inserta distalmente en la base de los metacarpos 2 y 3, el flexor carpi ulnaris de inserta en el pisiforme y de allí en el ganchoso y en la base del quinto metacarpiano.
El retináculo flexor es una banda fibrosa cuadrada que va medialmente del pisiforme al gancho del ganchoso y lateralmente del escafoides al borde del trapecio. El pisiforme es visible, así como el tubérculo del escafoides.

\subsubsection{Tabaquera anatómica}
Es una depresión en la cara lateral de la muñeca que se acentúa al extender el pulgar. Está limitada anteriormente por los tendones del abductor largo del pulgar y del extensor corto del pulgar y posteriormente por el extensor largo del pulgar. 
El abductor largo del pulgar se inserta distal al primer metacarpiano, el extensor corto del pulgar en la falange proximal del pulgar y el extensor largo del pulgar en la falange distal del primer dedo. La vena cefálica cruza la fosa superficialmente.

\subsubsection{Visón dorsal de la muñeca y de la mano}
La epífisis distal del radio es visible. Las articulaciones metacarpofalángicas de segunda a cuarta son facilmente visibles. El primer metecarpiano es mucho más móvil que los otros y su articulación carpometacarpiana es facilmente palpable (visible no mucho). Las articulaciones interfalángicas son palpables alrededor de su circunferencia.
Los tendones del extensor de los dedos están sujetos al hueso mediante el retináculo extensor que pasa por encima oblícuo desde la cara lateral del radio hasta la porcion medial del cúbito. Cada tendón del extensor de los dedos forma un triángulo (la expansión dorsal) sobre la articulación metacarpofalángica. Los músculos lumbricales e interóseos correspondientes se insertan en la base de esta expansión. Otros dos tendones pueden verse en el dorso de la muñeca, el extensor de meñique que discurre lateral al tendón largo del quinto dedo y el extensor del índice que discurre medial al tendón largo del dedo índice. Ambos insertan en la expansión dorsal.

\subsubsection{Movimientos de la muñeca y de la mano}
La flexión se realiza en la articulación mediocarpal (entre las dos hileras de carpos) desarrollada por los músculos flexor carpi radialis y flexor carpi ulnaris ayudados por los flexores largos de los dedos. La extensión se realiza en la articulación radiocarpal por los músculos extensor carpi radialis longus y brevis y el extensor carpi ulnaris ayudados por los extensores largos de los dedos. La abducción es en la articulación mediocarpal por los músculos flexor carpi radialis y extensor carpi radialis longus y brevis. La adducción en la articulación radiocarpal mediante el flexor carpi ulnaris y el extensor carpi ulnaris.

\subsubsection{Eminencia tenar}
Es el abultamiento lateral de la palma de la mano y está formada por los músculos cortos del pulgar: los músculos abductor, floxor y oponente del pulgar que se insertan en el tubérculo del escafoides, la cresta del trapecio y el retináculo flexor. El abductor corto y el flexor corto del pulgar insertan en la falange proximal del pulgar y el oponente del pulgar en el primer metacarpiano.
La flexión se combina con la rotación interna y la realizan los flexores largo y corto y el oponente del pulgar. La extensión se combina con la rotación externa y la realizan los extensores largo y corto y abductor largo del pulgar. EL abductor corto del pulgar realiza la abducción, el adductor del pulgar la adducción y el oponente del pulgar la oposición.

\subsubsection{Eminencia hipotenar}
Está formada por el abductor, el flexor y el oponente del meñique, originándose en el retináculo flexor hasta la falange proximal y el margen cubital del quinto metacarpiano. El meñique, aunque menos móvil que el pulgar, puede realizar una discreta rotación.

\subsubsection{Músculos interóseos}
Las articulaciones metacarpofalángicas de 2º a 4º pueden realizar abducción y aduccion (las del pulgar solo flexión y extensión). Movimientos que se deben a la acción de los músculos interóseos ventrales (adducción) y dorsales (abducción). Los interóseos palmares se insertan en las caras palmares de los metacarpianos 1º, 2º, 4º y 5º. Los dorsales son más largos y potentes originándose por dos cabezas en los lados de los metacarpianos adyacentes. Todos se insertan en la falange proximal y en la expansión del extensor correspondiente.

\subsubsection{Músculos lumbricales}
Son cuatro músculos débiles que se originan en la cara lateral de de cada tendón del flexor profundo de los dedos en la palma y se insertan en la cara lateral de de la expansión dorsal del mismo dedo. Flexionan la articulación metacarpofalángica y extienden las articulaciones interfalángicas proximal y distal, ayudados por los músculos interóseos.

La extensión de los dedos la lleva a cabo el músculo extensor largo de los dedos, ayudado por el extensor del índice y del meñique.

Colores: primarios, secundarios, fríos, cálidos, paleta de colores.

Línea, dibujo

Luz

Perspectiva

Movimiento o quietud

Proporciones

Iconografía

Resolución del tema

Momento de la narración elegido
