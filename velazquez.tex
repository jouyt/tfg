\section{Análisis de la obra pictórica: Cristo crucificado de Velázquez } 

\begin{figure}[ht!]
    \centering
    \includegraphics[width=1.0\textwidth]{velazquez.jpg}
    \caption{Cristo crucificado de Velázquez. %© Madrid, Museo Nacional del Prado} %. URL: www.museodelprado.es/coleccion/galeria-on-line/galeria-on-line/zoom/1/obra/cristo-crucificado-1/oimg/0/}
\end{figure}

\newpage

%Arte, estilo: Barroco
%Cronología: 1632
%Lugar: Museo del Prado, Madrid, España
%Autor: Diego Velázquez
%Título: Cristo crucificado

%Función: Aunque se encontraba en el Convento de la encarnación de San Plácido, no está claro donde estuvo los primeros años tras su realización ni la función específica que desarrollaba. Posteriormente se tiene constancia de que estuvo en la sacristía de este convento hasta alrededor del 1804, año en el que Godoy lo compra al convento. Pasa después por distintas manos, llegando finalmente al Museo de Prado en 1829, donde permanece actualmente.
% El Cristo crucificado de velazquez, trasfondo histórico religioso. 
% http://www.museodelprado.es/coleccion/galeria-on-line/galeria-on-line/obra/cristo-crucificado-1/

\begin{description}
\item[Cronología:] 1632
\item[Lugar:] Museo del Prado, Madrid
\item[Estilo:] Barroco
\item[Técnica:] Óleo sobre lienzo
%\item[Autor] Diego Rodríguez de Silva y Velázquez
%\item[Título] Cristo crucificado
\end{description}

\textbf{Contexto histórico:}

La realización de esta obra tuvo lugar en la atmósfera religiosa y social de la contrarreforma. Durante este período la Iglesia católica intentaba consolidar su poder bajo la amenaza de la reforma protestante, es por eso que durante esta época se produjeron un gran número de obras de contenido religioso como esta de Velázquez.

En esta obra Velázquez aplicó los conocimientos que adquirió en su viaje a Italia, los cuales se pueden observar en la realización estructurada y detallada de la anatomía de Cristo.

Aunque en las primeras representaciones de la crucifixión de Cristo éste estaba clavado mediante cuatro clavos, a partir del siglo XIII comienzan a ser más habituales los tres clavos. Sin embargo, Velázquez en su obra no siguió el estilo de su eṕoca, sino las recomendaciones de su maestro Francisco Pacheco en cuanto al número de clavos que mantienen a la figura clavada en la cruz. Por ello podemos observar los pies de Cristo apoyados en un subpedaneum, cuya utilización se cree posterior a la época de Cristo, al que se ensamblan mediante dos clavos. También podemos apreciar los clavos de las manos colocados entre los metacarpos (entre el segundo y el tercero o entre el tercero y el cuarto), según la interpretación bíblica de la época, contrariamente a la creencia actual principal de que se encontraban a la altura de las muñecas.

Además, empleó los contrastes de luz de acuerdo a la corriente tenebrista, siguiendo a Caravaggio, el inventor de esta técnica referente a la iluminación. Por ello vemos grandes contrastes entre la luz y las sombras de unas zonas del cuerpo a otras, como si la luz procediera de un foco fijo, que en el caso de este cuadro provendría de la zona superior izquierda.

La oscuridad que rodea al Cristo, sin ningún espacio ni persona a su alrededor, hace que la escena produzca una sensación mayor de soledad. Este tipo de composición es frecuente durante el siglo XVII, en contraposición a las obras de los siglos anteriores que solían mostrar la crucifixión de Cristo en una atmósfera totalmente distinta, con multitud de figuras a su alrededor.
% http://www.arteespana.com/velazquez.htm

\vspace{12pt}
\textbf{Anatomía de superficie:}

En esta obra Velázquez nos cautiva con una figura con una postura serena y de proporciones y anatomía estudiadas. %Responde a un modelo de siete cabezas y media\footnote{Aunque para calcular las distintas proporciones del cuerpo humano se han utilizado diferentes medidas anatómicas, el canon más utilizado ha sido el que propone hacerlo mediante la medida de la cabeza. Existen tres cánones para determinar las proporciones de la figura humana: el cánon de siete cabezas y media, el cánon más realista en cuanto a medidas, que se considera la figura común, el cánon de ocho cabezas considerado la figura ideal y el cánon de ocho cabezas y media, con el que se representan las figuras heróicas.}. Su longitud con los brazos extendidos es igual a su altura y el ángulo que forman entre ellos es de 115º.

El autor, no se centra demasiado en la reproducción de la muerte de la figura. De hecho, aunque la cabeza de la figura caiga sobre su pecho, pudiendo sugerir la muerte de éste, tal y como apreciamos por la músculatura de la espalda, brazos y piernas da la impresión de que todavía se encuentra con vida. Se podría pensar que en lugar de haber fallecido, el Cristo representado se encuentra en una posición de descanso tras el sufrimiento experimentado antes de la crucifixión. Los brazos se encuentran extendidos a ambos lados del cuerpo y los músculos del brazo y de la espalda que contribuyen a ello están claramente contraídos y extendidos, contrariamente a lo que ocurriría si Cristo ya hubiese perecido, en cuyo caso la musculatura permanecería laxa y exánime.

Por la posición de las piernas, podemos apreciar que, al tener ambos pies apoyados, carga el peso sobre el lado derecho, inclinando la cadera hacia este mismo lado, y flexionando la articulación de la rodilla contraria, es también evidente que Cristo no ha fallecido aún. El autor utiliza este ligerísimo contrapposto clásico \footnote{Posición asimétrica característica de buena parte de las esculturas griegas y romanas, en que el peso del cuerpo descansa principalmente en una pierna, por lo que la cadera correspondiente se eleva respecto de la otra. %http://www.arts4x.com/spa/d/contrapposto/contrapposto.htm
} como elemento embellecedor de la obra dejando de lado la indudable incapacidad de una persona crucificada y fallecida, o al borde del fallecimiento, para realizar esta acción.

Además, la figura no muestra la lividez típica de un cadáver, todo lo contrario, la figura posee el color de una persona totalmente sana, a pesar de la pérdida de sangre que Cristo debió experimentar durante su calvario. Otro detalle importante de la obra es la escasa sangre que emana de las heridas de Cristo: apenas varias gotas en las heridas de los clavos y aún menos en la herida de lanza que muestra en el costado derecho. Un rasgo de la obra que  manifiesta la muerte de Cristo, que como ya se ha dicho para nada está representada, es precisamente esta herida del costado, la cual, en teoría, fue realizada postmortem, como verificación de la muerte.

No se contempla tampoco la posibilidad de que la figura se encuentre en estado de \textit{rigor mortis}, ya que éste se produce tras una relajación total de los músculos, por muy breve que pueda ser, tras la muerte del individuo. Así, una persona que muere en la cruz no adoptaría esta posición ante la rigidez del \textit{rigor mortis} después de la relajación de los músculos: las estremidades inferiores no estarían estiradas, sino que serían las extremidades superiores las que, clavadas en la cruz, sujetarían el cuerpo muerto.

Una serie de músculos colaboran en la consecución de la postura en la que se encuentra la figura a analizar.

La espalda se encuentra erguida, los músculos que se encargan de mantener esta postura son los principales erectores de la columna: los músculos del grupo iliocostal, los del grupo longuísimo y los del espinoso.

El músculo dorsal ancho, que se puede apreciar en la obra, proviene de la espalda e inserta en el ángulo inferior de la escápula. Su tendón estrecho envuelve al músculo redondo mayor que forma el pliegue posterior de la axila. Ambos músculos junto con el pectoral mayor, el que además forma el pliegue anterior de la axila, elevan el tronco cuando los brazos se encuentran fijos, como es el caso de la figura examinada.

El músculo trapecio, cuya principal función es la elevación de la escápula, también se observa. El serrato anterior gira la escápula y, por tanto, junto con el músculo trapecio y el elevador de la escápula que la elevan hacen que la cavidad glenoidea se oriente hacia arriba y adelante, como en la figura de la obra de Velázquez.
El deltoides, que se ve fácilmente redondeando y sustentando la articulación del hombro por su parte superior, es el principal abductor del brazo, ya que sigue el movimiento abductor que el músculo supraespinoso inicia, por lo que tiene gran relevancia en la posición de crucifixión en la que ambos brazos se encuentran en posición de abducción.

En el brazo se observan tanto el bíceps como el tríceps. El bíceps forma una prominencia en la parte anterior del brazo y se encarga de la supinación del antebrazo, la cual realiza con la ayuda de los músculos supinadores. El tríceps por su parte se encarga de la extensión de la articulación del codo. En este movimiento le secunda el músculo ancóneo, que no es visible en la figura de la obra analizada por encontrarse en la parte posterior del codo.

La mano se encuentra en posición de reposo en la que las articulaciones falángicas y metacarpofalángicas se encuentran ligeramente flexionadas. El músculo palmar largo contribuye sutilmente a la flexión de las articulaciones metacarpofalángicas, que es realizada fundamentalmente por los músculos flexores de los dedos.

Al tratarse de un individuo delgado y favorecido por la postura en que se encuentra (brazos estirados hacia los lados y hacia arriba) se puede intuir la caja torácica perfectamente. El borde costal inferior formado por las seis últimas costillas es fácilmente visible, al igual que la depresión en la que se encuentra el cuerpo del esternón y el contorno de algunas costillas (probablemente la cuarta, quinta, sexta y séptima).

En el abdomen se encuentran los músculos abdominales, que se pueden apreciar, aunque no muy marcados. Estos músculos forman una vaina fibrosa a cada lado de la línea media. Ambas vainas se unen en esta, formando la línea alba. También se divisa, bastante bien, la línea semilunar, que marca el borde lateral del músculo recto del abdomen cuya principal función en la figura es el mantenimiento de la postura erecta apoyando así a los músculos erectores de la columna.

%El músculo mas superficial de los músculos laminares del abdomen es el oblicuo externo. Éste, entre la espina y la tuberosidad púbica, gira sobre sí mismo y forma el ligamento inguinal. La depresión que suele haber a la altura de ese ligamento, y que suele ser más visible en los hombres, se puede observar perfectamente en la obra pictórica.

En la posición de las piernas colaboran varios músculos. La rodilla derecha de la figura se encuentra en posición de extensión, mientras que la otra se encuentra ligeramente flexionada. Los encargados de la extensión son el cuadriceps y los músculos del tracto iliotibial, y los de la flexión son los músculos poplíteos y los gemelos. Al estar de pie soportando el peso sobre una pierna, el lado de la pelvis que soporta el peso se eleva %. Esta acción está realizada por los glúteos medio y menor 
y deja a la vista la espina ilíaca del lado derecho.
%La rótula o patella se puede apreciar, al igual que la tuberosidad tibial, sobre todo en la rodilla flexionada.

Los pies se encuentran apoyados en una tabla, soportando el peso del cuerpo, y se ubican en paralelo entre ellos ligeramente separados anteriormente, con un clavo en cada dorso y sangre que emana de la herida. %En la parte anterior de éstos, los dedos se encuentran relajados.

%El Cristo representado se encuentra crucificado en una cruz \textit{commisa} en cuya parte superior se ubica el cartel en el que se indica que es el Rey de los Judíos.

%En el cuadro se puede observar la parte izquierda de la cara, la parte derecha está cubierta por el cabello que cae sobre ella. En cuanto a aquello que podemos ver, se trata de una cara inexpresiva con barba que cubre parte de las mejillas y el mentón. Bajo esta barba se puede apreciar la forma del hueso de la manbíbula (cuerpo, rama y ángulo). La mucosa de los labios es apenas visible entre el vello, aunque sí se intuyen incluso la comisura labial izquierda. 

%La nariz se presenta en el centro de la cara de forma piramidal. La estructura de la nariz la proporcionan los huesos nasales que articulan entre sí, con el hueso frontal(por donde se une a la frente) y con el hueso maxilar (por el que se une  a las mejillas). La parte más prominente de la nariz es el cartílago nasal que separa las dos fosas nasales. Se aprecia el espacio del surco naso-labial, aunque éste no se ve directamente, porque al igual que el mentón, está cubierto de vello. Sin embargo, sí se distingue la aleta nasal izquierda y el septo nasal, a la vez que se puede intuir la narina izquierda de la nariz. Se puede también vislumbrar una pequeña arruga, la línea naso-labial. Se ve la prominencia de la mejilla, formada por el hueso cigomático. 

%Entre la nariz y la frente se encuentra una depresión denominada nasión. La frente, formada por la convexidad lisa del hueso frontal, apenas es visible puesto que está tapada por la corona de espinas. Sus bordes inferiores (arcos supraciliares) dan lugar al borde superior de cada órbita. Además estos dos arcos se unen mediante un puente denomindo gabela. Superficialmente a estas elevaciones se encuentran las cejas.

%Medialmente el globo ocular está limitado por los huesos maxilar y frontal que articulan entre ellos, lateralmente está limitado por los huesos frontal y cigomático e inferiormente por el hueso maxilar. El globo ocular izquierdo no es visible, solo se adivina debajo del párpado superior que lo cubre, en el que además se puede observar una pequeña arruga. Sí se puede vislumbrar las uniones medial y lateral entre los párpados superior e inferior, que limitarían la hendidura palpebral en caso de estar abiertos. En sus bordes se encuentran las pestañas, manifestadas en la obra mediante una línea más oscura en el borde de los párpados.

%Ambas orejas están tapadas por el cabello, por lo que sus partes no son visibles.

%El resto de la cabeza está formada por la bóveda craneal, la cual se insinua debajo del cabello. La cara lateral de la bóveda la componen los huesos frontal, parietal, occipital y temporal.
%El cuello no es visible, al estar la cabeza inclinada hacia abajo, ésta tapa las estructuras anatómicas del cuello.

%\vspace{12pt}

%En la imagen se puede observar el torso desnudo de Cristo crucificado, en el que se aprecian una serie de estructuras anatómicas. 

%Al tratarse de un individuo delgado se puede intuir la caja torácico perfectamente. Vislumbramos, a duras penas, la clavícula izquierda que articula con el esternón (manubrio) y la primera costilla, la derecha, no es visible, puesto que está tapada por el cabello. El borde costal inferior formado por las seis últimas costillas es fácilmente visible, al igual que la depresión en la que se encuentra el cuerpo del esternón y el contorno de algunas costillas (probablemente la cuarta, quinta, sexta y séptima), sobre todo las del lado derecho, ya que la sombra del propio cuadro nos  impide ver bien las del izquierdo. La escotadura esternal, también se puede intuir sobre la cara superior del manubrio, sin embargo, hay que dejarse llevar un poco por la imaginación, ya que, de nuevo la sombra del cuadro nos impide su correcta visualización. No se observa la apófisis xifoides, que está cubierta, por el recto anterior del abdomen, lo que se considera normal anatómicamente.

%En cuanto a los músculos podemos vislumbrar el pectoral mayor, que inserta en el borde interno de la clavícula, en los cartílagos de las seis costillas superiores y en el húmero, formando así el pliegue anterior de la axila, muy claro en la imagen, junto con el músculo deltoides, que se puede contemplar recubriendo la articulación del hombro. Lateralmente no se observan claramente los músculos. El serrato anterior que inserta en el borde anterior interno de la escápula, y en las ocho costillas superiores, junto con sus interdigitaciones con el músculo oblicuo externo podrían ser visibles en un individuo delgado y musculoso como el Cristo que representa este cuadro, sin embargo no es fácil de reconocer. Sí se puede apreciar el músculo dorsal ancho, proveniente de la espalda e insertado en el ángulo inferior de la escápula. Su tendón estrecho envuelve al músculo redondo mayor que forma el pliegue posterior de la axila. Este va desde el borde lateral de la escápula hasta el surco bicipital del húmero. El borde lateral de la escápula puede percibirse a través de la masa del redondo mayor.

%Los pezones se encuentran en el Cristo de la obra alrededor del cuarto espacio intercostal, lo que es anatómicamente correcto.

\vspace{12pt}

%En el abdomen, que delimita superiormente con el reborde costal, que como ya he dicho es claramente visible en esta obra, se encuentran los músculos abdominales, que se pueden apreciar, aunque no muy marcados. Estos músculos forman una vaina fibrosa a cada lado de la línea media. Ambas vainas se unen en esta, formando la línea alba. La figura no está suficientemente musculada para poder apreciar las tres inserciones tendinosas que se encontrarían encima del ombligo, a duras penas podríamos localizar una de ellas. Sí se divisa bastante bien la línea semilunar, que marca el borde lateral del músculo recto del abdomen.

%La espina ilíaca si se aprecia, sobre todo en el lado derecho, que es el lado hacia el que está ligeramente inclinada la cadera.

%El músculo mas superficial de los músculos laminares del abdomen es el oblicuo externo , pero como ya he comentado anteriormente no se ve claramente, y aunque sepamos que tiene que estar ahí, sus interdigitaciones con el serrato anterior no son visibles en el cuadro. Este músculo se inserta superiormente en las ocho costillas inferiores e inferiormente en la sínfisis de púbis, tuberosidad púbica, espina ilíaca anterosuperior y cresta ilíaca. Entre la espina y la tuberosidad púbica gira sobre sí mismo y forma el ligamento inguinal. La depresión que suele haber a la altura de ese ligamento, y que suele ser más visible en hombre, se puede observar perfectamente en la obra pictórica.

%\vspace{12pt}

%Las estructuras pélvicas no son visibles, debido a que el autor coloca encima un paño blanco con el que las tapa.

%\vspace{12pt}

%Las extemidades superiores se encuentran extendidas cada una hacia un costado de la cruz: Los músculos son visibles claramente.

%Antes de empezar con los músculos del brazo propiamente dichos, remarcaré algúno que se encuentra sustentando la articulación del hombro. El músculo trapecio sí se observa, este lateralmente se inserta en el tercio externo de la clavícula, el acromion y la espina de la escápula. Estas estructuras no son visibles anteriormente en la posición en la que se encuentra la figura analizada (brazos extendidos hacia los lados), sin embargo podemos apreciar un hueco en la unión de este músculo a las estructuras previamente citadas. El músculo que no nos deja ver esas estructuras es el deltoides, que también se ve fácilmente redondeando la articulación del hombro por su parte superior, cubriendo la epífisis proximal del húmero. Se inserta medialmente, al igual que el trapecio, en la clavícula, el acromion y la espina de la escápula y lateralmente en el tubérculo deltoideo de la cara lateral del húmero. Es el principal abductor del brazo por lo que tiene gran relevancia en esta posición.

%En el brazo se observan tanto el bíceps como el tríceps. El bíceps forma una prominencia en la parte anterior del brazo. Tiene dos cabezas, de las cuales la corta inserta en la apófisis coracoides y la larga en la cara superior de la fosa glenoidea de la escápula. Distalmente se inserta en la tuberosidad bicipital del radio, pasando por el medio de la fosa cubital.

%En ambos brazos podemos distinguir la fosa cubital. Esta está limitada superiormete por la línea que va desde la epitróclea hasta el epicóndilo del húmero, lateralmente por el músculo braquiorradial y medialmente por el pronador redondo.

%Los huesos de la articulación del codo suelen ser evidentes. En esta obra no se pueden distinguir claramente las distintas estructuras debido a la posición de los brazos (se verían mejor desde la cara posterior). Lo único que se aprecia de forma clara es la epitróclea del húmero.

%Los brazos se encuentran en total supinación y extensión. La extensión la realiza el tríceps junto con la contribución de los músculos del origen del extensor común. Y de la supinación se encargan el bíceps y el supinador.

%A la altura de las muñecas podemos observar las apófisis estiloides del cúbito y del radio, que en una persona un poco más gruesa no serían visibles. Además, se podrían intuir los tendones de los músculos del antebrazo que pasan a la mano. Cuando se flexiona la muñeca sobresalen tres tendones. El flexor carpi radialis se inserta distalmente en la base de los metacarpos 2 y 3, el flexor carpi ulnaris de inserta en el pisiforme y de allí en el ganchoso y en la base del quinto metacarpiano.

%Las venas, tanto cefálica como basílica suelen se visibles en mayor o menor medida en hombres jóvenes, delgados y en cierta medida musculados, no obstante, no han sido dibujadas por el artista en esta figura.

%En la obra se puede apreciar la palma de la mano derecha, aunque parcialmente, pues en ella se encuentra clavado el clavo a la cruz. Además los dedos se encuentran ligeramente flexionados, por lo que podemos vislumbrar las articulaciones interfalángicas en ellos. El metacarpiano del primer dedo también lo podemos distinguir en el cuadro. Este posee más movilidad que el resto de los metacarpianos, que además no son visibles en el cuadro, puesto que están tapados por los músculos de la palma.

%En la mano izquierda básicamente se observan las mismas estructuras que en la derecha. Pero además podemos percibir la denominada tabaquera anatómica. La tabaquera anatómica es una concavidad presente en la cara lateral de la muñeca, que se evidencia más al extender el primer dedo. Está limitada por varios tendones: el abductor largo del pulgar, que se inserta distal al primer metacarpiano; el extensor corto del pulgar, que se inserta en la falange proximal del pulgar; y el extensor largo del pulgar, que inserta en la falange distal del primer dedo.

%En las palmas de ambas manos se observan tanto la eminencia tenar, como la hipotenar.

%La emnencia tenar se denomina al abultamiento lateral presente en la palma de la mano, a la altura del primer dedo. Está formada por los músculos cortos del pulgar: los músculos abductor, flexor y oponente del pulgar.

%La eminencia hipotenar se denomina al abultamiento medial de la palma de la mano a la altura del quinto metecarpiano. Está formada por el abductor, el flexor y el oponente del meñique.
%Podemos apreciar cierta flexión en las articulaciones metacarpofalángicas, la cual es producida por los músculos lumbricales que se encuentran entre los dedos. Los dedos también mantienen cierta flexión, producida esta por los tendones de los flexores superficial y profundo de los dedos.

% \subsubsection{Visión anterior del antebrazo}
% El músculo pronador se inserta sobre la cara anterior de la epitróclea y en la región central del radio. Es un pronador del antebrazo. Los tendones de los músculos flexor carpi radialis, palmar mayor, flexor superficial de los dedos y flexor carpi ulnaris pasan a la mano. Son también flexores débiles del codo.
% El músculo braquiorradial se inserta proximalmente  en los dos tercios superiores de la cresta supracondílea del húmero y distalmete en la cara lateral de la epífisis distal del radio. Este músculo prona o supina el antebrazo.

%El pisiforme es visible, así como el tubérculo del escafoides.

%\vspace{12pt}

%Podemos apreciar claramente la masa muscular de las estremidades inferiores, ya que estas se encuentran al descubierto. Sin embargo no es así de fácil distinguir los músculos específicos que producen esta masa muscular.

%Podemos intuir a duras penas por donde pasa el músculo sartorio, que que inserta en la espina ilíaca y en la superficie subcutánea medial de la parte superior de la tibia. Así mismo se podría señalar el músculo cuadriceps, músculo principal de la cara anterior del muslo, junto con sus tres cabezas superficiales: vasto externo, vasto interno y recto anterior, no el vasto intermedio, que va bajo el recto anterior.

%Los músculos pectíneo, psoas-ilíaco y recto interno no son visibles pues están tapados por el paño blanco que también tapa las estructuras pélvicas. No son claramente apreciables otros músculos superficiales como el tensor de la fascia lata (que inserta en la cara externa del ilion y en la tibia, formando el tracto iliotibial al unirse con el glúteo mayor), que esta situado en el lateral del muslo, superficialmente al trocánter mayor del fémur.

%El músculo cuadriceps forma en su parte inferior el "ligamento patelar" que recubre la rótula y se inserta en la tuberosidad tibial. En una figura delgada como esta la rótula o patella se puede apreciar, al igual que la tuberosidad tibial, aunque son más fáciles de apreciar en la flexión de la rodilla.

%El resto de estructuras de la rodilla no son apreciables, aunque se pueden intuir: condilos femorales, ligamentos de la articulación, tubérculo adductor del fémur, retináculos rotulianos medial y lateral...

%La rodilla derecha de la figura se encuentran una en posición de extensión, mientras que la otra se encuentra ligeramente flexionada. Los músculos encargados de la extensión son: el cuadriceps y los músculos del tracto iliotibial, y los de la flexión son: los músculos poplíteos y los gemelos.

%Podemos apreciar la tibia desde la tuberosidad tibial hasta el maleolo medial, en su recorrido subcutáneo.

%La masa muscular que se puede visualizar la conforman varios músculos. El músculo tibial anterior se encuentra en el lateral de la pierna y es el único de los cuatro músculos que inserta en la tibia. Además tiene un tendón distal que pasa hacia la base del primer metatarsiano. Otros músculos que forman esta masa muscular son el extensor largo del dedo gordo y el extensor largo de los dedos. Sin embargo también se pueden apreciar en la cara lateral músculos que en principio forman parte de la parte posterior: el sóleo y una de las cabezas del gemelo (superior al sóleo), que se unen para formar el comunmente denominado tendón de aquiles. Más lateral se encuentra el músculo peroneus tertius.

%A la altura de la articulación del tobillo se pueden contemplar ambos maleolos, interno formado por la tibia y externo formado por el peroné (que en el resto de su recorrido por la pierna está rodeado por músculos). Anatómicamente el maleolo externo se encuentra un centímetro más abajo que el maleolo interno. En la representación pictórica se reproduce correctamente esta característica.

%Al estar de pie soportando el peso sobre una pierna, el lado de la pelvis que no soporta el peso se eleva. Esta acción está realizada por los glúteos medio y menor.

%Los pies se encuentran apoyados en una tabla,lo que permite que sus rodillas y cadera estén ligeramente flexionadas. Apoyados sobre el hueso calcáneo y la articulación metatarsofalángica, cuya prominencia es visible bajo la grasa subcutánea que sirve de almohadilla, dejan intuir claramente el arco plantar, mantenido por los músculos cortos y los tendones del flexor largo.

%Los pies se ubican en paralelo entre ellos ligeramente suparados aneriormente, con un clavo en cada dorso y sangre que emana de la herida, lo que nos hace más dificil la visión de las estructuras que se encuentran en el dorso del pie. La cara lateral es peor visible, debido a la gran dfuminación de los colores en esta zona y a la oscuridad con la que el artista simula la sombra de la figura. Más medialmente se pueden observar tendones (del músculo extensor largo de los dedos, del músculo extensor largo del dedo gordo y de los músculos peroneos) y prominencias óseas que no son claramente distinguibles. En la parte anterior, los dedos se encuentran relajados y las uñas nacen de cada uno de ellos superficialmente.

% El extensor corto de los dedos es el único músculo situado en el dorso del pie y se extiende desde el hueso calcáneo. Sus tendones se insertan en la falange proximal del primer dedo y en la cara lateral de los tendones del extensor de los dedos segundo, tercero y cuarto.

% La cara posteromedial del calcáneo, el sustentaculum tali, el tubérculo del navicular.


%Colores: primarios, secundarios, fríos, cálidos, paleta de colores.

%Línea, dibujo

%Luz

%Perspectiva

%Movimiento o quietud

%Proporciones

%Iconografía

%Resolución del tema

%Momento de la narración elegido
