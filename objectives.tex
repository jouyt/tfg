\section{Objetivos}
Los objetivos a cumplir en este trabajo son varios. Existe un objetivo principal y otros secundarios.

El objetivo principal será identificar y estudiar la anatomía de superficie en la muerte de Cristo. Para ello se utilizarán varias obras diferentes cuyo análisis nos mostrará la agonía y muerte de Cristo representadas en diferentes épocas que, influenciadas a su vez por el contexto social y religioso, influyen en la pintura o escultura, en la anatomía y en la visión de Cristo de cada época.

Otros objetivos secundarios, relacionados con éste, son: afianzar los conocimientos anatómicos y desarrollarlos, analizar la historia de la anatomía que nos ha permitido llegar al conocimiento anatómico actual e identificar los cambios que se producen en el cuerpo en la agonía y en el momento del fallecimiento y cómo lo han representado en las distintas obras que se analizarán.

Además existen varios objetivos transversales que tienen que ver con la adquisición y la consolidación de conocimientos y habilidades para la realización del Trabajo de Fin de Grado (TFG). Me refiero en específico a: la utilización de las bases de datos de investigación para conseguir información, el desarrollo de la capacidad de síntesis, el análisis crítico de la información encontrada y la elaboración adecuada y estructurada de un TFG.